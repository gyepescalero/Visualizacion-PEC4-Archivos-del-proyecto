\documentclass[11pt]{article}

    \usepackage[breakable]{tcolorbox}
    \usepackage{parskip} % Stop auto-indenting (to mimic markdown behaviour)
    
    \usepackage{iftex}
    \ifPDFTeX
    	\usepackage[T1]{fontenc}
    	\usepackage{mathpazo}
    \else
    	\usepackage{fontspec}
    \fi

    % Basic figure setup, for now with no caption control since it's done
    % automatically by Pandoc (which extracts ![](path) syntax from Markdown).
    \usepackage{graphicx}
    % Maintain compatibility with old templates. Remove in nbconvert 6.0
    \let\Oldincludegraphics\includegraphics
    % Ensure that by default, figures have no caption (until we provide a
    % proper Figure object with a Caption API and a way to capture that
    % in the conversion process - todo).
    \usepackage{caption}
    \DeclareCaptionFormat{nocaption}{}
    \captionsetup{format=nocaption,aboveskip=0pt,belowskip=0pt}

    \usepackage[Export]{adjustbox} % Used to constrain images to a maximum size
    \adjustboxset{max size={0.5\linewidth}{0.6\paperheight}}
    \usepackage{float}
    \floatplacement{figure}{H} % forces figures to be placed at the correct location
    \usepackage{xcolor} % Allow colors to be defined
    \usepackage{enumerate} % Needed for markdown enumerations to work
    \usepackage{geometry} % Used to adjust the document margins
    \usepackage{amsmath} % Equations
    \usepackage{amssymb} % Equations
    \usepackage{textcomp} % defines textquotesingle
    % Hack from http://tex.stackexchange.com/a/47451/13684:
    \AtBeginDocument{%
        \def\PYZsq{\textquotesingle}% Upright quotes in Pygmentized code
    }
    \usepackage{upquote} % Upright quotes for verbatim code
    \usepackage{eurosym} % defines \euro
    \usepackage[mathletters]{ucs} % Extended unicode (utf-8) support
    \usepackage{fancyvrb} % verbatim replacement that allows latex
    \usepackage{grffile} % extends the file name processing of package graphics 
                         % to support a larger range
    \makeatletter % fix for grffile with XeLaTeX
    \def\Gread@@xetex#1{%
      \IfFileExists{"\Gin@base".bb}%
      {\Gread@eps{\Gin@base.bb}}%
      {\Gread@@xetex@aux#1}%
    }
    \makeatother

    % The hyperref package gives us a pdf with properly built
    % internal navigation ('pdf bookmarks' for the table of contents,
    % internal cross-reference links, web links for URLs, etc.)
    \usepackage{hyperref}
    % The default LaTeX title has an obnoxious amount of whitespace. By default,
    % titling removes some of it. It also provides customization options.
    \usepackage{titling}
    \usepackage{longtable} % longtable support required by pandoc >1.10
    \usepackage{booktabs}  % table support for pandoc > 1.12.2
    \usepackage[inline]{enumitem} % IRkernel/repr support (it uses the enumerate* environment)
    \usepackage[normalem]{ulem} % ulem is needed to support strikethroughs (\sout)
                                % normalem makes italics be italics, not underlines
    \usepackage{mathrsfs}
    

    
    % Colors for the hyperref package
    \definecolor{urlcolor}{rgb}{0,0,0}
    \definecolor{linkcolor}{rgb}{0,0,0}
    \definecolor{citecolor}{rgb}{0,0,0}

    % ANSI colors
    \definecolor{ansi-black}{HTML}{3E424D}
    \definecolor{ansi-black-intense}{HTML}{282C36}
    \definecolor{ansi-red}{HTML}{E75C58}
    \definecolor{ansi-red-intense}{HTML}{B22B31}
    \definecolor{ansi-green}{HTML}{00A250}
    \definecolor{ansi-green-intense}{HTML}{007427}
    \definecolor{ansi-yellow}{HTML}{DDB62B}
    \definecolor{ansi-yellow-intense}{HTML}{B27D12}
    \definecolor{ansi-blue}{HTML}{208FFB}
    \definecolor{ansi-blue-intense}{HTML}{0065CA}
    \definecolor{ansi-magenta}{HTML}{D160C4}
    \definecolor{ansi-magenta-intense}{HTML}{A03196}
    \definecolor{ansi-cyan}{HTML}{60C6C8}
    \definecolor{ansi-cyan-intense}{HTML}{258F8F}
    \definecolor{ansi-white}{HTML}{C5C1B4}
    \definecolor{ansi-white-intense}{HTML}{A1A6B2}
    \definecolor{ansi-default-inverse-fg}{HTML}{FFFFFF}
    \definecolor{ansi-default-inverse-bg}{HTML}{000000}

    % commands and environments needed by pandoc snippets
    % extracted from the output of `pandoc -s`
    \providecommand{\tightlist}{%
      \setlength{\itemsep}{0pt}\setlength{\parskip}{0pt}}
    \DefineVerbatimEnvironment{Highlighting}{Verbatim}{commandchars=\\\{\}}
    % Add ',fontsize=\small' for more characters per line
    \newenvironment{Shaded}{}{}
    \newcommand{\KeywordTok}[1]{\textcolor[rgb]{0.00,0.44,0.13}{\textbf{{#1}}}}
    \newcommand{\DataTypeTok}[1]{\textcolor[rgb]{0.56,0.13,0.00}{{#1}}}
    \newcommand{\DecValTok}[1]{\textcolor[rgb]{0.25,0.63,0.44}{{#1}}}
    \newcommand{\BaseNTok}[1]{\textcolor[rgb]{0.25,0.63,0.44}{{#1}}}
    \newcommand{\FloatTok}[1]{\textcolor[rgb]{0.25,0.63,0.44}{{#1}}}
    \newcommand{\CharTok}[1]{\textcolor[rgb]{0.25,0.44,0.63}{{#1}}}
    \newcommand{\StringTok}[1]{\textcolor[rgb]{0.25,0.44,0.63}{{#1}}}
    \newcommand{\CommentTok}[1]{\textcolor[rgb]{0.38,0.63,0.69}{\textit{{#1}}}}
    \newcommand{\OtherTok}[1]{\textcolor[rgb]{0.00,0.44,0.13}{{#1}}}
    \newcommand{\AlertTok}[1]{\textcolor[rgb]{1.00,0.00,0.00}{\textbf{{#1}}}}
    \newcommand{\FunctionTok}[1]{\textcolor[rgb]{0.02,0.16,0.49}{{#1}}}
    \newcommand{\RegionMarkerTok}[1]{{#1}}
    \newcommand{\ErrorTok}[1]{\textcolor[rgb]{1.00,0.00,0.00}{\textbf{{#1}}}}
    \newcommand{\NormalTok}[1]{{#1}}
    
    % Additional commands for more recent versions of Pandoc
    \newcommand{\ConstantTok}[1]{\textcolor[rgb]{0.53,0.00,0.00}{{#1}}}
    \newcommand{\SpecialCharTok}[1]{\textcolor[rgb]{0.25,0.44,0.63}{{#1}}}
    \newcommand{\VerbatimStringTok}[1]{\textcolor[rgb]{0.25,0.44,0.63}{{#1}}}
    \newcommand{\SpecialStringTok}[1]{\textcolor[rgb]{0.73,0.40,0.53}{{#1}}}
    \newcommand{\ImportTok}[1]{{#1}}
    \newcommand{\DocumentationTok}[1]{\textcolor[rgb]{0.73,0.13,0.13}{\textit{{#1}}}}
    \newcommand{\AnnotationTok}[1]{\textcolor[rgb]{0.38,0.63,0.69}{\textbf{\textit{{#1}}}}}
    \newcommand{\CommentVarTok}[1]{\textcolor[rgb]{0.38,0.63,0.69}{\textbf{\textit{{#1}}}}}
    \newcommand{\VariableTok}[1]{\textcolor[rgb]{0.10,0.09,0.49}{{#1}}}
    \newcommand{\ControlFlowTok}[1]{\textcolor[rgb]{0.00,0.44,0.13}{\textbf{{#1}}}}
    \newcommand{\OperatorTok}[1]{\textcolor[rgb]{0.40,0.40,0.40}{{#1}}}
    \newcommand{\BuiltInTok}[1]{{#1}}
    \newcommand{\ExtensionTok}[1]{{#1}}
    \newcommand{\PreprocessorTok}[1]{\textcolor[rgb]{0.74,0.48,0.00}{{#1}}}
    \newcommand{\AttributeTok}[1]{\textcolor[rgb]{0.49,0.56,0.16}{{#1}}}
    \newcommand{\InformationTok}[1]{\textcolor[rgb]{0.38,0.63,0.69}{\textbf{\textit{{#1}}}}}
    \newcommand{\WarningTok}[1]{\textcolor[rgb]{0.38,0.63,0.69}{\textbf{\textit{{#1}}}}}
    
    
    % Define a nice break command that doesn't care if a line doesn't already
    % exist.
    \def\br{\hspace*{\fill} \\* }
    % Math Jax compatibility definitions
    \def\gt{>}
    \def\lt{<}
    \let\Oldtex\TeX
    \let\Oldlatex\LaTeX
    \renewcommand{\TeX}{\textrm{\Oldtex}}
    \renewcommand{\LaTeX}{\textrm{\Oldlatex}}
    % Document parameters
    % Document title
    \title{Untitled1}
    
    
    
    
    
% Pygments definitions
\makeatletter
\def\PY@reset{\let\PY@it=\relax \let\PY@bf=\relax%
    \let\PY@ul=\relax \let\PY@tc=\relax%
    \let\PY@bc=\relax \let\PY@ff=\relax}
\def\PY@tok#1{\csname PY@tok@#1\endcsname}
\def\PY@toks#1+{\ifx\relax#1\empty\else%
    \PY@tok{#1}\expandafter\PY@toks\fi}
\def\PY@do#1{\PY@bc{\PY@tc{\PY@ul{%
    \PY@it{\PY@bf{\PY@ff{#1}}}}}}}
\def\PY#1#2{\PY@reset\PY@toks#1+\relax+\PY@do{#2}}

\expandafter\def\csname PY@tok@w\endcsname{\def\PY@tc##1{\textcolor[rgb]{0.73,0.73,0.73}{##1}}}
\expandafter\def\csname PY@tok@c\endcsname{\let\PY@it=\textit\def\PY@tc##1{\textcolor[rgb]{0.25,0.50,0.50}{##1}}}
\expandafter\def\csname PY@tok@cp\endcsname{\def\PY@tc##1{\textcolor[rgb]{0.74,0.48,0.00}{##1}}}
\expandafter\def\csname PY@tok@k\endcsname{\let\PY@bf=\textbf\def\PY@tc##1{\textcolor[rgb]{0.00,0.50,0.00}{##1}}}
\expandafter\def\csname PY@tok@kp\endcsname{\def\PY@tc##1{\textcolor[rgb]{0.00,0.50,0.00}{##1}}}
\expandafter\def\csname PY@tok@kt\endcsname{\def\PY@tc##1{\textcolor[rgb]{0.69,0.00,0.25}{##1}}}
\expandafter\def\csname PY@tok@o\endcsname{\def\PY@tc##1{\textcolor[rgb]{0.40,0.40,0.40}{##1}}}
\expandafter\def\csname PY@tok@ow\endcsname{\let\PY@bf=\textbf\def\PY@tc##1{\textcolor[rgb]{0.67,0.13,1.00}{##1}}}
\expandafter\def\csname PY@tok@nb\endcsname{\def\PY@tc##1{\textcolor[rgb]{0.00,0.50,0.00}{##1}}}
\expandafter\def\csname PY@tok@nf\endcsname{\def\PY@tc##1{\textcolor[rgb]{0.00,0.00,1.00}{##1}}}
\expandafter\def\csname PY@tok@nc\endcsname{\let\PY@bf=\textbf\def\PY@tc##1{\textcolor[rgb]{0.00,0.00,1.00}{##1}}}
\expandafter\def\csname PY@tok@nn\endcsname{\let\PY@bf=\textbf\def\PY@tc##1{\textcolor[rgb]{0.00,0.00,1.00}{##1}}}
\expandafter\def\csname PY@tok@ne\endcsname{\let\PY@bf=\textbf\def\PY@tc##1{\textcolor[rgb]{0.82,0.25,0.23}{##1}}}
\expandafter\def\csname PY@tok@nv\endcsname{\def\PY@tc##1{\textcolor[rgb]{0.10,0.09,0.49}{##1}}}
\expandafter\def\csname PY@tok@no\endcsname{\def\PY@tc##1{\textcolor[rgb]{0.53,0.00,0.00}{##1}}}
\expandafter\def\csname PY@tok@nl\endcsname{\def\PY@tc##1{\textcolor[rgb]{0.63,0.63,0.00}{##1}}}
\expandafter\def\csname PY@tok@ni\endcsname{\let\PY@bf=\textbf\def\PY@tc##1{\textcolor[rgb]{0.60,0.60,0.60}{##1}}}
\expandafter\def\csname PY@tok@na\endcsname{\def\PY@tc##1{\textcolor[rgb]{0.49,0.56,0.16}{##1}}}
\expandafter\def\csname PY@tok@nt\endcsname{\let\PY@bf=\textbf\def\PY@tc##1{\textcolor[rgb]{0.00,0.50,0.00}{##1}}}
\expandafter\def\csname PY@tok@nd\endcsname{\def\PY@tc##1{\textcolor[rgb]{0.67,0.13,1.00}{##1}}}
\expandafter\def\csname PY@tok@s\endcsname{\def\PY@tc##1{\textcolor[rgb]{0.73,0.13,0.13}{##1}}}
\expandafter\def\csname PY@tok@sd\endcsname{\let\PY@it=\textit\def\PY@tc##1{\textcolor[rgb]{0.73,0.13,0.13}{##1}}}
\expandafter\def\csname PY@tok@si\endcsname{\let\PY@bf=\textbf\def\PY@tc##1{\textcolor[rgb]{0.73,0.40,0.53}{##1}}}
\expandafter\def\csname PY@tok@se\endcsname{\let\PY@bf=\textbf\def\PY@tc##1{\textcolor[rgb]{0.73,0.40,0.13}{##1}}}
\expandafter\def\csname PY@tok@sr\endcsname{\def\PY@tc##1{\textcolor[rgb]{0.73,0.40,0.53}{##1}}}
\expandafter\def\csname PY@tok@ss\endcsname{\def\PY@tc##1{\textcolor[rgb]{0.10,0.09,0.49}{##1}}}
\expandafter\def\csname PY@tok@sx\endcsname{\def\PY@tc##1{\textcolor[rgb]{0.00,0.50,0.00}{##1}}}
\expandafter\def\csname PY@tok@m\endcsname{\def\PY@tc##1{\textcolor[rgb]{0.40,0.40,0.40}{##1}}}
\expandafter\def\csname PY@tok@gh\endcsname{\let\PY@bf=\textbf\def\PY@tc##1{\textcolor[rgb]{0.00,0.00,0.50}{##1}}}
\expandafter\def\csname PY@tok@gu\endcsname{\let\PY@bf=\textbf\def\PY@tc##1{\textcolor[rgb]{0.50,0.00,0.50}{##1}}}
\expandafter\def\csname PY@tok@gd\endcsname{\def\PY@tc##1{\textcolor[rgb]{0.63,0.00,0.00}{##1}}}
\expandafter\def\csname PY@tok@gi\endcsname{\def\PY@tc##1{\textcolor[rgb]{0.00,0.63,0.00}{##1}}}
\expandafter\def\csname PY@tok@gr\endcsname{\def\PY@tc##1{\textcolor[rgb]{1.00,0.00,0.00}{##1}}}
\expandafter\def\csname PY@tok@ge\endcsname{\let\PY@it=\textit}
\expandafter\def\csname PY@tok@gs\endcsname{\let\PY@bf=\textbf}
\expandafter\def\csname PY@tok@gp\endcsname{\let\PY@bf=\textbf\def\PY@tc##1{\textcolor[rgb]{0.00,0.00,0.50}{##1}}}
\expandafter\def\csname PY@tok@go\endcsname{\def\PY@tc##1{\textcolor[rgb]{0.53,0.53,0.53}{##1}}}
\expandafter\def\csname PY@tok@gt\endcsname{\def\PY@tc##1{\textcolor[rgb]{0.00,0.27,0.87}{##1}}}
\expandafter\def\csname PY@tok@err\endcsname{\def\PY@bc##1{\setlength{\fboxsep}{0pt}\fcolorbox[rgb]{1.00,0.00,0.00}{1,1,1}{\strut ##1}}}
\expandafter\def\csname PY@tok@kc\endcsname{\let\PY@bf=\textbf\def\PY@tc##1{\textcolor[rgb]{0.00,0.50,0.00}{##1}}}
\expandafter\def\csname PY@tok@kd\endcsname{\let\PY@bf=\textbf\def\PY@tc##1{\textcolor[rgb]{0.00,0.50,0.00}{##1}}}
\expandafter\def\csname PY@tok@kn\endcsname{\let\PY@bf=\textbf\def\PY@tc##1{\textcolor[rgb]{0.00,0.50,0.00}{##1}}}
\expandafter\def\csname PY@tok@kr\endcsname{\let\PY@bf=\textbf\def\PY@tc##1{\textcolor[rgb]{0.00,0.50,0.00}{##1}}}
\expandafter\def\csname PY@tok@bp\endcsname{\def\PY@tc##1{\textcolor[rgb]{0.00,0.50,0.00}{##1}}}
\expandafter\def\csname PY@tok@fm\endcsname{\def\PY@tc##1{\textcolor[rgb]{0.00,0.00,1.00}{##1}}}
\expandafter\def\csname PY@tok@vc\endcsname{\def\PY@tc##1{\textcolor[rgb]{0.10,0.09,0.49}{##1}}}
\expandafter\def\csname PY@tok@vg\endcsname{\def\PY@tc##1{\textcolor[rgb]{0.10,0.09,0.49}{##1}}}
\expandafter\def\csname PY@tok@vi\endcsname{\def\PY@tc##1{\textcolor[rgb]{0.10,0.09,0.49}{##1}}}
\expandafter\def\csname PY@tok@vm\endcsname{\def\PY@tc##1{\textcolor[rgb]{0.10,0.09,0.49}{##1}}}
\expandafter\def\csname PY@tok@sa\endcsname{\def\PY@tc##1{\textcolor[rgb]{0.73,0.13,0.13}{##1}}}
\expandafter\def\csname PY@tok@sb\endcsname{\def\PY@tc##1{\textcolor[rgb]{0.73,0.13,0.13}{##1}}}
\expandafter\def\csname PY@tok@sc\endcsname{\def\PY@tc##1{\textcolor[rgb]{0.73,0.13,0.13}{##1}}}
\expandafter\def\csname PY@tok@dl\endcsname{\def\PY@tc##1{\textcolor[rgb]{0.73,0.13,0.13}{##1}}}
\expandafter\def\csname PY@tok@s2\endcsname{\def\PY@tc##1{\textcolor[rgb]{0.73,0.13,0.13}{##1}}}
\expandafter\def\csname PY@tok@sh\endcsname{\def\PY@tc##1{\textcolor[rgb]{0.73,0.13,0.13}{##1}}}
\expandafter\def\csname PY@tok@s1\endcsname{\def\PY@tc##1{\textcolor[rgb]{0.73,0.13,0.13}{##1}}}
\expandafter\def\csname PY@tok@mb\endcsname{\def\PY@tc##1{\textcolor[rgb]{0.40,0.40,0.40}{##1}}}
\expandafter\def\csname PY@tok@mf\endcsname{\def\PY@tc##1{\textcolor[rgb]{0.40,0.40,0.40}{##1}}}
\expandafter\def\csname PY@tok@mh\endcsname{\def\PY@tc##1{\textcolor[rgb]{0.40,0.40,0.40}{##1}}}
\expandafter\def\csname PY@tok@mi\endcsname{\def\PY@tc##1{\textcolor[rgb]{0.40,0.40,0.40}{##1}}}
\expandafter\def\csname PY@tok@il\endcsname{\def\PY@tc##1{\textcolor[rgb]{0.40,0.40,0.40}{##1}}}
\expandafter\def\csname PY@tok@mo\endcsname{\def\PY@tc##1{\textcolor[rgb]{0.40,0.40,0.40}{##1}}}
\expandafter\def\csname PY@tok@ch\endcsname{\let\PY@it=\textit\def\PY@tc##1{\textcolor[rgb]{0.25,0.50,0.50}{##1}}}
\expandafter\def\csname PY@tok@cm\endcsname{\let\PY@it=\textit\def\PY@tc##1{\textcolor[rgb]{0.25,0.50,0.50}{##1}}}
\expandafter\def\csname PY@tok@cpf\endcsname{\let\PY@it=\textit\def\PY@tc##1{\textcolor[rgb]{0.25,0.50,0.50}{##1}}}
\expandafter\def\csname PY@tok@c1\endcsname{\let\PY@it=\textit\def\PY@tc##1{\textcolor[rgb]{0.25,0.50,0.50}{##1}}}
\expandafter\def\csname PY@tok@cs\endcsname{\let\PY@it=\textit\def\PY@tc##1{\textcolor[rgb]{0.25,0.50,0.50}{##1}}}

\def\PYZbs{\char`\\}
\def\PYZus{\char`\_}
\def\PYZob{\char`\{}
\def\PYZcb{\char`\}}
\def\PYZca{\char`\^}
\def\PYZam{\char`\&}
\def\PYZlt{\char`\<}
\def\PYZgt{\char`\>}
\def\PYZsh{\char`\#}
\def\PYZpc{\char`\%}
\def\PYZdl{\char`\$}
\def\PYZhy{\char`\-}
\def\PYZsq{\char`\'}
\def\PYZdq{\char`\"}
\def\PYZti{\char`\~}
% for compatibility with earlier versions
\def\PYZat{@}
\def\PYZlb{[}
\def\PYZrb{]}
\makeatother


    % For linebreaks inside Verbatim environment from package fancyvrb. 
    \makeatletter
        \newbox\Wrappedcontinuationbox 
        \newbox\Wrappedvisiblespacebox 
        \newcommand*\Wrappedvisiblespace {\textcolor{red}{\textvisiblespace}} 
        \newcommand*\Wrappedcontinuationsymbol {\textcolor{red}{\llap{\tiny$\m@th\hookrightarrow$}}} 
        \newcommand*\Wrappedcontinuationindent {3ex } 
        \newcommand*\Wrappedafterbreak {\kern\Wrappedcontinuationindent\copy\Wrappedcontinuationbox} 
        % Take advantage of the already applied Pygments mark-up to insert 
        % potential linebreaks for TeX processing. 
        %        {, <, #, %, $, ' and ": go to next line. 
        %        _, }, ^, &, >, - and ~: stay at end of broken line. 
        % Use of \textquotesingle for straight quote. 
        \newcommand*\Wrappedbreaksatspecials {% 
            \def\PYGZus{\discretionary{\char`\_}{\Wrappedafterbreak}{\char`\_}}% 
            \def\PYGZob{\discretionary{}{\Wrappedafterbreak\char`\{}{\char`\{}}% 
            \def\PYGZcb{\discretionary{\char`\}}{\Wrappedafterbreak}{\char`\}}}% 
            \def\PYGZca{\discretionary{\char`\^}{\Wrappedafterbreak}{\char`\^}}% 
            \def\PYGZam{\discretionary{\char`\&}{\Wrappedafterbreak}{\char`\&}}% 
            \def\PYGZlt{\discretionary{}{\Wrappedafterbreak\char`\<}{\char`\<}}% 
            \def\PYGZgt{\discretionary{\char`\>}{\Wrappedafterbreak}{\char`\>}}% 
            \def\PYGZsh{\discretionary{}{\Wrappedafterbreak\char`\#}{\char`\#}}% 
            \def\PYGZpc{\discretionary{}{\Wrappedafterbreak\char`\%}{\char`\%}}% 
            \def\PYGZdl{\discretionary{}{\Wrappedafterbreak\char`\$}{\char`\$}}% 
            \def\PYGZhy{\discretionary{\char`\-}{\Wrappedafterbreak}{\char`\-}}% 
            \def\PYGZsq{\discretionary{}{\Wrappedafterbreak\textquotesingle}{\textquotesingle}}% 
            \def\PYGZdq{\discretionary{}{\Wrappedafterbreak\char`\"}{\char`\"}}% 
            \def\PYGZti{\discretionary{\char`\~}{\Wrappedafterbreak}{\char`\~}}% 
        } 
        % Some characters . , ; ? ! / are not pygmentized. 
        % This macro makes them "active" and they will insert potential linebreaks 
        \newcommand*\Wrappedbreaksatpunct {% 
            \lccode`\~`\.\lowercase{\def~}{\discretionary{\hbox{\char`\.}}{\Wrappedafterbreak}{\hbox{\char`\.}}}% 
            \lccode`\~`\,\lowercase{\def~}{\discretionary{\hbox{\char`\,}}{\Wrappedafterbreak}{\hbox{\char`\,}}}% 
            \lccode`\~`\;\lowercase{\def~}{\discretionary{\hbox{\char`\;}}{\Wrappedafterbreak}{\hbox{\char`\;}}}% 
            \lccode`\~`\:\lowercase{\def~}{\discretionary{\hbox{\char`\:}}{\Wrappedafterbreak}{\hbox{\char`\:}}}% 
            \lccode`\~`\?\lowercase{\def~}{\discretionary{\hbox{\char`\?}}{\Wrappedafterbreak}{\hbox{\char`\?}}}% 
            \lccode`\~`\!\lowercase{\def~}{\discretionary{\hbox{\char`\!}}{\Wrappedafterbreak}{\hbox{\char`\!}}}% 
            \lccode`\~`\/\lowercase{\def~}{\discretionary{\hbox{\char`\/}}{\Wrappedafterbreak}{\hbox{\char`\/}}}% 
            \catcode`\.\active
            \catcode`\,\active 
            \catcode`\;\active
            \catcode`\:\active
            \catcode`\?\active
            \catcode`\!\active
            \catcode`\/\active 
            \lccode`\~`\~ 	
        }
    \makeatother

    \let\OriginalVerbatim=\Verbatim
    \makeatletter
    \renewcommand{\Verbatim}[1][1]{%
        %\parskip\z@skip
        \sbox\Wrappedcontinuationbox {\Wrappedcontinuationsymbol}%
        \sbox\Wrappedvisiblespacebox {\FV@SetupFont\Wrappedvisiblespace}%
        \def\FancyVerbFormatLine ##1{\hsize\linewidth
            \vtop{\raggedright\hyphenpenalty\z@\exhyphenpenalty\z@
                \doublehyphendemerits\z@\finalhyphendemerits\z@
                \strut ##1\strut}%
        }%
        % If the linebreak is at a space, the latter will be displayed as visible
        % space at end of first line, and a continuation symbol starts next line.
        % Stretch/shrink are however usually zero for typewriter font.
        \def\FV@Space {%
            \nobreak\hskip\z@ plus\fontdimen3\font minus\fontdimen4\font
            \discretionary{\copy\Wrappedvisiblespacebox}{\Wrappedafterbreak}
            {\kern\fontdimen2\font}%
        }%
        
        % Allow breaks at special characters using \PYG... macros.
        \Wrappedbreaksatspecials
        % Breaks at punctuation characters . , ; ? ! and / need catcode=\active 	
        \OriginalVerbatim[#1,codes*=\Wrappedbreaksatpunct]%
    }
    \makeatother

    % Exact colors from NB
    \definecolor{incolor}{HTML}{303F9F}
    \definecolor{outcolor}{HTML}{D84315}
    \definecolor{cellborder}{HTML}{CFCFCF}
    \definecolor{cellbackground}{HTML}{F7F7F7}
    
    % prompt
    \makeatletter
    \newcommand{\boxspacing}{\kern\kvtcb@left@rule\kern\kvtcb@boxsep}
    \makeatother
    \newcommand{\prompt}[4]{
        \ttfamily\llap{{\color{#2}[#3]:\hspace{3pt}#4}}\vspace{-\baselineskip}
    }
    

    
    % Prevent overflowing lines due to hard-to-break entities
    \sloppy 
    % Setup hyperref package
    \hypersetup{
      breaklinks=true,  % so long urls are correctly broken across lines
      colorlinks=true,
      urlcolor=urlcolor,
      linkcolor=linkcolor,
      citecolor=citecolor,
      }
    % Slightly bigger margins than the latex defaults
    
    \geometry{verbose,tmargin=1in,bmargin=1in,lmargin=1in,rmargin=1in}
\author{Germán Yepes Calero}
\title{Pactos de paz contra la extinción de las poblaciones indígenas
\\ \large Visualización de datos}

\date{1 de Junio, 2020}



\setlength{\parskip}{2mm}

\begin{document}
\maketitle

\pagebreak

{
  \hypersetup{linkcolor=black}
  \tableofcontents
}

\pagebreak 

\section{Introducción}

En la presente práctica se exponen todos los informes y memorias dirigidos a la creación de una visualización basada en pactos de paz extraídos del proyecto \emph{Peace Agreements} \cite{peace}. Los pactos irán relacionados con acuerdos que incumban poblaciones o tribus indígenas.

El proyecto se inició en la PEC2: "Hipótesis y exploración inicial de datos" (\ref{PEC2}), en la que describíamos y analizábamos 
en líneas generales los pactos, infiriendo conclusiones y comportamientos interesantes, los cuales poder incluir posteriormente en la visualización.

A continuación, en la PEC3: "Informe de proyecto" (\ref{PEC3}), introducíamos el proyecto de visualización de datos con el conocimiento de los datos adquirido en la PEC2. En esta fase describimos los objetivos e intereses de la visualización, así como un esquema orientativo del diseño que queríamos crear. 

Por último, incluiremos la parte relativa a la PEC4: "Creación de la visualización" (\ref{PEC4}), incorporando la visualización definitiva de nuestro dataset, las etapas y herramientas implicadas en su elaboración y un análisis de la consecución de objetivos que habíamos planteado en la PEC3.

\subsection{Título del proyecto}

El título del proyecto es descriptivo y simple: "Pactos de paz contra la extinción de las poblaciones indígenas" y, como su propio nombre indica, hace referencia al conjunto de tratados con el que trataremos a lo largo del proyecto, todos ellos relacionados en algún aspecto con poblaciones indígenas. Además, hace hincapié sobre lo que intenta concienciar, la dura situación de estas poblaciones y su importancia cultural.

\subsection{Descripción del proyecto}

El propósito del proyecto será elaborar una visualización que ayude a llamar la atención sobre la cuestión de las poblaciones indígenas. A lo largo de las tres PECs incluidas en este informe, trataremos de extraer conclusiones y curiosidades de la base de datos para su posterior implementación en una visualización que ayude a concienciar al usuario de la importancia de preservar la cultura, religión, forma de vida... de estas minorías. Además de intentar hacer comprender las difíciles condiciones en las que se encuentran y el peligro que corren si no las protegemos. En todo momento nos centraremos en el eje central del proyecto, la visualización, intentando sacar a relucir de la forma más gráfica y llamativa posible la información que alberga nuestra base de datos.

\subsection{URL de la visualización}

La visualización fue elaborada mediante la aplicación instalada \emph{Tableau} \cite{tableau} y posteriormente
subida y guardada en la plataforma online \emph{Tableau Public} \cite{public}. Su enlace URL es el siguiente:
\url{https://public.tableau.com/profile/germ.n.yepes#!/vizhome/Acuerdosdepazparalaspoblacionesindgenas_Visualizacin/Story1?publish=yes}

A parte de la visualización, también se podrá acceder a los archivos con el código usado en Tableau (.twb), los programas Jupyter (.ipynb) o el texto editado en Overleaf ( .tex). Todos estos archivos se incluirán en un proyecto GitHub \cite{github} con enlace: 
\url{https://github.com/gyepescalero/Visualizacion-PEC4-Archivos-del-proyecto}

Además aparecerán las informes de las PECs anteriores.

\section{PEC2: Hipótesis y exploración inicial de datos}\label{PEC2}

\subsection{Descripción}

El objetivo principal de la PEC2 fue analizar una selección de tratados de paz extraídos del projecto \emph{Peace Agreements} de la
Universidad de Edimburgo (peaceagreements.org). El proyecto recoge un
gran número de acuerdos de paz de diversa índole, firmados desde 1990
hasta la actualidad. En nuestro caso, abarcaremos únicamente aquellos
tratados en los que participen o se tengan en cuenta a comunidades de
poblaciones indígenas.

El procedimiento a seguir será el siguiente. Primero, expondremos la
base de datos seleccionada y su estructura. Continuaremos escogiendo y
describiendo los atributos más relevantes para, posteriormente, proceder
a su análisis a través de múltiples herramientas de visualización.
Durante esta fase de análisis, trataremos de extraer conclusiones sobre
la base de datos: qué tipo de acuerdos son los más comunes, diferencias
entre las distintas sociedades, comportamientos interesantes, relaciones entre los tipos de acuerdos...
Finalmente, resumiremos lo aprendido en la exploración de los datos exponiendo de manera conjunta todas las visualizaciones.

La totalidad de la práctica, redacción, programación y representación de
visualizaciones, será llevada a cabo con lenguaje Latex en Overleaf y lenguaje de
programación Python en Jupyter. A lo largo del apartado se expondrán con
más detalle las librerías y herramientas utilizadas.

\pagebreak

\subsection{Presentación de los
datos}

Comenzamos haciendo una evaluación inicial de nuestros datos. Para ello,
importaremos el dataframe con la librería pandas (pandas.pydata.org) y
mostraremos las cinco primeras líneas correspondientes a cinco tratados
de paz donde se tienen en cuenta las poblaciones indígenas.

    \begin{tcolorbox}[breakable, size=fbox, boxrule=1pt, pad at break*=1mm,colback=cellbackground, colframe=cellborder]
\prompt{In}{incolor}{276}{\boxspacing}
\begin{Verbatim}[commandchars=\\\{\}]
\PY{k+kn}{import} \PY{n+nn}{pandas} \PY{k}{as} \PY{n+nn}{pd}
\PY{n}{data} \PY{o}{=} \PY{n}{pd}\PY{o}{.}\PY{n}{read\PYZus{}csv}\PY{p}{(}\PY{l+s+s1}{\PYZsq{}}\PY{l+s+s1}{Indigenous.csv}\PY{l+s+s1}{\PYZsq{}}\PY{p}{)}
\PY{n}{data}\PY{o}{.}\PY{n}{head}\PY{p}{(}\PY{p}{)}
\end{Verbatim}
\end{tcolorbox}

    \begin{center}
    \adjustimage{min size={0.28\linewidth}{0.28\paperheight}}{headIndi.png}
    \end{center}
    { \hspace*{\fill} \\}
        
    En la primera columna "Con", podemos ver el país que llevó a cabo el
acuerdo y en algunos tratados, como el de Bangaldesh con Chittagong Hill
Tracts (línea 0), el nombre de la población indígena que también participó en el pacto. En otros tratados, como los de la República
Central Africana (líneas 1 y 2) y los de Colombia (líneas 3 y 4),
únicamente aparece el nombre del país. En estos casos, podemos ver
quiénes integran la otra parte del acuerdo en la cuarta columna "PPName"
y sexta columna "Agt", donde se nos indica el nombre del proceso de paz
y el nombre del acuerdo. También se nos incluyen números identificadores
para cada tratado, así como su fecha.

En la tabla solo vemos las primeras columnas, utilizadas para
identificar los acuerdos. Veamos ahora el número total de columnas de
nuestra base de datos o, lo que es lo mismo, cuantos atributos presenta
de cada pacto. También queremos conocer el número total de tratados que
constituyen el dataframe es decir, el número total de filas del mismo.
Utilizaremos la librería numpy (numpy.org)

    \begin{tcolorbox}[breakable, size=fbox, boxrule=1pt, pad at break*=1mm,colback=cellbackground, colframe=cellborder]
\prompt{In}{incolor}{277}{\boxspacing}
\begin{Verbatim}[commandchars=\\\{\}]
\PY{k+kn}{import} \PY{n+nn}{numpy} \PY{k}{as} \PY{n+nn}{np}
\PY{n+nb}{print}\PY{p}{(}\PY{l+s+s2}{\PYZdq{}}\PY{l+s+s2}{Número de columnas:}\PY{l+s+s2}{\PYZdq{}}\PY{p}{,} \PY{n+nb}{len}\PY{p}{(}\PY{p}{(}\PY{n}{data}\PY{o}{.}\PY{n}{columns}\PY{p}{)}\PY{p}{)}\PY{p}{)}
\PY{n+nb}{print}\PY{p}{(}\PY{l+s+s2}{\PYZdq{}}\PY{l+s+s2}{Número de filas:}\PY{l+s+s2}{\PYZdq{}}\PY{p}{,} \PY{n}{np}\PY{o}{.}\PY{n}{shape}\PY{p}{(}\PY{n}{data}\PY{p}{)}\PY{p}{[}\PY{l+m+mi}{0}\PY{p}{]}\PY{p}{)}
\end{Verbatim}
\end{tcolorbox}

    \begin{Verbatim}[commandchars=\\\{\}]
Número de columnas: 265
Número de filas: 116
    \end{Verbatim}

    Tenemos 265 metadatos o atributos y 116 pactos en nuestra base de
datos. El enorme número de atributos nos impide que podamos analizar
todos ellos. Por esta razón, seleccionaremos aquellos que puedan
proporcionar mejores resultados en la extracción de conclusiones. La
mayoría de atributos son numéricos e indican el nivel de repercusión del
tratado en un ámbito concreto. El resto de atributos serán de tipo
categórico y clasificarán los acuerdos según sus características
generales. Los atributos categóricos escogidos en este apartado son los
siguientes:

\begin{itemize}
\tightlist
\item
  Con: Países/regiones/poblaciones que han llevado a cabo el acuerdo.
\item
  Reg: Región del mundo en la que el conflicto resuelto por el acuerdo
  ha tenido lugar. Tenemos las siguientes regiones:

  \begin{itemize}
  \tightlist
  \item
    Africa (excl. MENA)
  \item
    Americas
  \item
    Asia and Pacific
  \item
    Europe and Eurasia
  \item
    Middle East and North Africa
  \item
    Cross-regional
  \item
    Other
  \end{itemize}
\item
  Dat: Fecha en la que el acuerdo fue firmado o negociado. Formato:
  AAAA-MM-DD
\item
  Contp: Tipo de conflicto que resuelve el acuerdo. Tenemos los
  siguientes tipos:

  \begin{itemize}
  \tightlist
  \item
    Government: Político o ideológico
  \item
    Territory: Territorial
  \item
    Government/territory: Político/territorial
  \item
    Inter-group: Conflictos entre grupos no estatales.
  \item
    Other: Otros tipos de acuerdo
  \end{itemize}
\item
  Agtp: Naturaleza primaria del acuerdo y el conflicto. Tenemos las
  siguientes naturalezas de acuerdo:

  \begin{itemize}
  \item
    Inter: Internacional
  \item
    Intra: Intranacional
  \item
    IntraLocal: Acuerdos relacionados con un conflicto intraestatal,
    pero con el objetivo de resolver problemas locales.
  \end{itemize}
\item
  Status: Estado del acuerdo. Variable categórica con cuatro tipos de estado de
  acuerdo distintos:

  \begin{itemize}
  \tightlist
  \item
    Multiparty signed/agreed: Acuerdo firmado o claramente aceptado por
    más de un grupo de participantes opuestos (pero no necesariamente
    todos por los grupos opuestos).
  \item
    Unilateral agreement: Acuerdo unilateral producido por una de las partes del conflicto
    pero en respuesta a un acuerdo con la otra parte.
  \item
    Status unclear: Si no está claro qué parte del acuerdo se firmó o
    quién acordó, pero hay alguna indicación de la documentación
    circundante de que fue firmado o acordado.
  \item
    Agreement with Subsequent Status: Acuerdo propuesto basado en el
    diálogo entre las partes y una "suposición" de lo que acordarán que,
    aunque no fue aceptado por las partes como un acuerdo, se convirtió en
    la base para desarrollos posteriores.
  \end{itemize}
\item
  Stage: Fase en la que se encuentra el acuerdo. Variable categórica con
  siete fases:

  \begin{itemize}
  \tightlist
  \item
    Pre: Proceso de prenegociación.
  \item
    SubPar: Acuerdo parcial en algunos aspectos del acuerdo. Algunas
    partes del acuerdo siguen pendientes de discusión.
  \item
    SubComp: Similar a la etapa SubPar pero existe un intento real de
    resolver el conflicto.
  \item
    Imp: Implementación o renegociación con el objetivo de implementar
    un acuerdo anterior.
  \item
    Ren: Renovación con acuerdos cortos que buscan renovar los
    compromisos anteriores.
  \item
    Cea: Fase de alto el fuego o relacionadas. Acuerdos que compromenten
    total o parcialmente al alto el fuego o similares.
  \item
    Other: Otros tipos de acuerdos que se encuentran en estados
    distintos a los anteriores.
  \end{itemize}
\end{itemize}

En cuanto a los atributos numéricos, nos centraremos principalmente en los
que aparecen un mayor número de veces en nuestro conjunto de acuerdos.
La selección aparece explicada con detalle en el subapartado de Exploración de
los datos. Por lo general, los atributos numéricos aparecen expresados con un
número del 0 al 3. Si el acuerdo no contiene referencias al ámbito
especificado o similares, el valor se establece en 0. Si existen tales
referencias, la variable toma los valores de la siguiente manera:

\begin{itemize}
\tightlist
\item
  1: Solo se menciona el ámbito sin especificar detalles y este no tiene
  relevancia real en el acuerdo.
\item
  2: El acuerdo contiene medidas que hacen referencia al ámbito
  especificado o similares, y estas brindan más detalles sobre el modo
  de implementación de las medidas.
\item
  3: El acuerdo dispone de medidas sustantivas respecto
  al ámbito especificado, dando detalles e indicando compromiso con su
  implementación.
\end{itemize}

Los atributos escogidos de este tipo son los siguientes:

\begin{itemize}
\item
  GRa: Razas/etnias/grupos nacionalistas. Señala las menciones en el
  acuerdo de raza, grupos étnicos, minorías nacionales, clanes u
  organizaciones "tribales" similares.
\item
  GRe: Grupos religiosos. Hace referencia a cualquier mención a grupos
  religiosos, ya sea en términos de esos grupos o en términos de la
  inclusión de religiones. ~
\item
  GRef: Refugiados. Hace referencia a los refugiados y a las personas
  desplazadas (incluida la repatriación) que aparecen en el acuerdo de
  paz.
\item
  TjVic: Justicia transicional. Incluye cualquier disposición que brinde
  asistencia específica a las víctimas del conflicto o medidas que guarden alguna relación con aquellos que han sufrido el conflicto, incluidas víctimas desaparecidas
\end{itemize}

También tenemos un gran número de variables binarias, las cuales toman
el valor de 1 si el acuerdo de paz aborda el tema especificado en el
atributo. Si en cambio el tema no se aborda en absoluto en el acuerdo
de paz, el valor en esta variable es 0. Únicamente escogeremos un atributo de este tipo:

\begin{itemize}
\item
  HrGen: Derechos humanos/Estado de derecho. Variable binaria, toma el
  valor 1 si el acuerdo de paz incluye referencias a los derechos
  humanos o al derecho internacional. Si no aparecen tales disposiciones
  en el acuerdo, el valor de esta variable es 0.
\end{itemize}

El resto de atributos vienen descritos en el "manual" de la web Peace Agreements.

Manual de atributos: \url{https://www.peaceagreements.org/files/PA-X%20codebook%20Version3.pdf}

    Antes de proceder al análisis queremos conocer las características
generales del conjunto de pactos de paz con el que estamos trabajando. Para
hacernos una mejor idea de como están distribuidos los tratados y de qué
tipo son, visualizaremos en histogramas las siete variables categóricas
que hemos considerado más relevantes. Las librerías utilizadas para
crear las visualizaciones serán matplotlib (matplotlib.org) y seaborn
(seaborn.pydata.org). Añadimos el código con el que realizamos este tipo de gráficas sns.catplot.

    \begin{tcolorbox}[breakable, size=fbox, boxrule=1pt, pad at break*=1mm,colback=cellbackground, colframe=cellborder]
\prompt{In}{incolor}{406}{\boxspacing}
\begin{Verbatim}[commandchars=\\\{\}]
\PY{n}{sns}\PY{o}{.}\PY{n}{set}\PY{p}{(}\PY{n}{font\PYZus{}scale}\PY{o}{=}\PY{l+m+mf}{1.2}\PY{p}{)}
\PY{n}{country\PYZus{}plot} \PY{o}{=} \PY{n}{sns}\PY{o}{.}\PY{n}{catplot}\PY{p}{(}\PY{n}{x}\PY{o}{=}\PY{l+s+s2}{\PYZdq{}}\PY{l+s+s2}{Con}\PY{l+s+s2}{\PYZdq{}}\PY{p}{,} \PY{n}{kind}\PY{o}{=}\PY{l+s+s2}{\PYZdq{}}\PY{l+s+s2}{count}\PY{l+s+s2}{\PYZdq{}}\PY{p}{,} \PY{n}{palette}\PY{o}{=}\PY{l+s+s2}{\PYZdq{}}\PY{l+s+s2}{cubehelix}\PY{l+s+s2}{\PYZdq{}}\PY{p}{,} \PY{n}{data}\PY{o}{=}\PY{n}{data}\PY{p}{,} \PY{n}{height}\PY{o}{=}\PY{l+m+mi}{5}\PY{p}{)}
\PY{n}{country\PYZus{}plot}\PY{o}{.}\PY{n}{set\PYZus{}axis\PYZus{}labels}\PY{p}{(} \PY{l+s+s2}{\PYZdq{}}\PY{l+s+s2}{País/Población indígena}\PY{l+s+s2}{\PYZdq{}}\PY{p}{,} \PY{l+s+s2}{\PYZdq{}}\PY{l+s+s2}{Número de acuerdos}\PY{l+s+s2}{\PYZdq{}}\PY{p}{)}
\PY{n}{country\PYZus{}plot}\PY{o}{.}\PY{n}{fig}\PY{o}{.}\PY{n}{set\PYZus{}size\PYZus{}inches}\PY{p}{(}\PY{l+m+mi}{30}\PY{p}{,}\PY{l+m+mi}{8}\PY{p}{)}
\PY{n}{country\PYZus{}plot}\PY{o}{.}\PY{n}{set\PYZus{}xticklabels}\PY{p}{(}\PY{n}{rotation}\PY{o}{=}\PY{l+m+mi}{90}\PY{p}{)}
\PY{n}{plt}\PY{o}{.}\PY{n}{title}\PY{p}{(}\PY{l+s+s2}{\PYZdq{}}\PY{l+s+s2}{Partes implicadas en el acuerdo}\PY{l+s+s2}{\PYZdq{}}\PY{p}{)}
\end{Verbatim}
\end{tcolorbox}

    \begin{center}
    \adjustimage{max size={0.5\linewidth}{0.6\paperheight}}{output_10_1.png}
    \end{center}
    { \hspace*{\fill} \\}
    
    Observando el histograma vemos como la mayor parte de tratados
relacionados con poblaciones indígenas incluye países como Filipinas,
Colombia, Guatemala, Nepal y México. También cabe destacar a la India,
que aparece en múltiples ocasiones pero con tratados que repercuten a
zonas de poblaciones indígenas distintas.

    \begin{center}
    \adjustimage{min size={0.21\linewidth}{0.21\paperheight}}{output_12_1.png}
    \end{center}
    { \hspace*{\fill} \\}
    
    Como era de esperar viendo los resultados del histograma anterior, la
mayoría de acuerdos tienen lugar en las regiones de Asia y el Pacífico
(India, Nepal, Filipinas...) y en América (Colombia, México...). También
tendrán relevancia en África.

    Para encontrar distribución de los acuerdos con poblaciones indígenas a
lo largo de los años, realizaremos la primera transformación a nuestra
base de datos. Nuestro objetivo será obtener una fila que indique
únicamente el valor del año en el que se produjo el acuerdo para poder
representar el número de acuerdos que se produjeron cada año.

    \begin{center}
    \adjustimage{min size={0.21\linewidth}{0.21\paperheight}}{output_15_1.png}
    \end{center}
    { \hspace*{\fill} \\}
    
    La interpretación de la gráfica nos revela que existen algunas
variaciones en el número de acuerdos elaborados cada año. Destaca por
encima de todos ellos 1996, con un total de 14 acuerdos, 5 por encima de
la segunda mayor marca registrada, la de 2006. Se trata de un valor extremo
o outlier que nos hace preguntarnos si ese año tuvo lugar algún
conflicto o suceso que pudo haber desencadenado tantos pactos.
Representemos los acuerdos que tuvieron lugar ese año.

    \begin{center}
    \adjustimage{min size={0.3\linewidth}{0.3\paperheight}}{output_17_1.png}
    \end{center}
    { \hspace*{\fill} \\}
    
    En el histograma podemos ver como los acuerdos con poblaciones indígenas
se dispararon ese año en Guatemala y México. Busquemos las razones que
expliquen el por qué de tan elevado número de acuerdos.

Por un lado, en Guatemala tenemos el acuerdo "Memoria del Silencio",
elaborado a finales de la Guerra Civil de Guatemala (1960-1996). El
principal motivo del tratado fue el genocidio guatemalteco, ocurrido
entre 1981 y 1983.

Fuente: \url{https://es.wikipedia.org/wiki/Genocidio\_guatemalteco}

Por otro lado, en México tenemos "Los Acuerdos de San Andrés sobre
Derechos y Cultura Indígena", un documento que el gobierno de México
firmó con el Ejército Zapatista de Liberación Nacional el 16 de febrero
de 1996, para comprometerse a incluir autonomía de los pueblos indígenas
de México.

Fuente: \url{https://es.wikipedia.org/wiki/Acuerdos\_de\_San\_Andr\%C3\%A9s}

    \begin{center}
    \adjustimage{max size={0.8\linewidth}{0.6\paperheight}}{output_19_1.png}
    \end{center}
    { \hspace*{\fill} \\}
    
    En cuanto al tipo de conflicto, trataremos esencialmente con conflictos
de tipo político. También tendremos que en parte importante de los acuerdos
también habrá un factor territorial importante. Podemos deducir que los
acuerdos estarán principalmente relacionados con el reconocimiento
político de las poblaciones indígenas y la conservación de sus
territorios.
        
    \begin{center}
    \adjustimage{max size={0.8\linewidth}{0.6\paperheight}}{output_21_1.png}
    \end{center}
    { \hspace*{\fill} \\}
    
    Como cabía esperar, la inmensa mayoría de conflictos tendrán lugar entre
la población indígena y el país con el que intentan llegar a un acuerdo.
Por lo tanto, serán conflictos de naturaleza intranacional y
generalmente no tendrán repercusión fuera del país.
        
    \begin{center}
    \adjustimage{max size={0.8\linewidth}{0.6\paperheight}}{output_23_1.png}
    \end{center}
    { \hspace*{\fill} \\}
    
    En este gráfica podemos apreciar la mayor parte de los acuerdos han sido
firmados y acordados por grupos pertenecientes a ambas partes del
conflicto. En el conjunto total tendremos un número muy pequeño de
acuerdos unilaterales o de estado incierto.
        
    \begin{center}
    \adjustimage{max size={0.8\linewidth}{0.6\paperheight}}{output_25_1.png}
    \end{center}
    { \hspace*{\fill} \\}
    
    En este caso si habrá una distribución repartida entre las distintas
fases en las que se encuentra el acuerdo. Muchos de los tratados se
hallarán en fase de acuerdo parcial ("SubPar"), seguidos por tratados en
fases de prenegociación, acuerdo parcial comprensivo ("SubComp") y de
implementación o negociación. En menor medida tendremos tratados en
fases de alto el fuego y de renovación.

\subsection{Exploración de los datos}

Una vez comprendemos un poco mejor la selección de tratados con la que
estamos trabajando, vamos a proceder a elegir las variables numéricas
con las que realizaremos un análisis más exhaustivo.

A la hora de escoger, hemos decidido elegir aquellas que tuvieran un mayor
peso y que, por lo tanto, apareciesen más veces y tomasen valores más
altos en los acuerdos. Para ello, hemos elaborado el siguiente código,
donde como resultado aparecen los atributos con mayor
relevancia en el cómputo general de todos los acuerdos.

    \begin{tcolorbox}[breakable, size=fbox, boxrule=1pt, pad at break*=1mm,colback=cellbackground, colframe=cellborder]
\prompt{In}{incolor}{412}{\boxspacing}
\begin{Verbatim}[commandchars=\\\{\}]
\PY{n}{int\PYZus{}attribute}\PY{o}{=}\PY{p}{[}\PY{p}{]}
\PY{k}{for} \PY{n}{i} \PY{o+ow}{in} \PY{n}{data}\PY{o}{.}\PY{n}{columns}\PY{p}{:}
    \PY{k}{if} \PY{n}{data}\PY{p}{[}\PY{n}{i}\PY{p}{]}\PY{o}{.}\PY{n}{dtypes}\PY{o}{==}\PY{l+s+s2}{\PYZdq{}}\PY{l+s+s2}{int64}\PY{l+s+s2}{\PYZdq{}} \PY{o+ow}{and} \PY{n}{data}\PY{p}{[}\PY{n}{i}\PY{p}{]}\PY{o}{.}\PY{n}{sum}\PY{p}{(}\PY{p}{)}\PY{o}{\PYZgt{}}\PY{l+m+mi}{50}\PY{p}{:}
        \PY{n}{int\PYZus{}attribute}\PY{o}{.}\PY{n}{append}\PY{p}{(}\PY{n}{i}\PY{p}{)}
\PY{n+nb}{print}\PY{p}{(}\PY{n}{int\PYZus{}attribute}\PY{p}{)}
\end{Verbatim}
\end{tcolorbox}

    \begin{Verbatim}[commandchars=\\\{\}]
['PP', 'AgtId', 'Lgt', 'N\_characters', 'Loc1GWNO', 'GCh', 'GDis', 'GRa', 'GRe',
'GInd', 'GIndRhet', 'GIndSubs', 'GOth', 'GRef', 'GSoc', 'GeWom', 'StDef', 'Pol',
'Cons', 'Ele', 'Civso', 'Pubad', 'Polps', 'Terps', 'Eps', 'HrGen', 'EqGen',
'HrDem', 'Prot', 'HrFra', 'Med', 'HrCit', 'Dev', 'DevSoc', 'Bus', 'Tax', 'Ce',
'SsrPol', 'SsrArm', 'SsrDdr', 'SsrPsf', 'TjAm', 'TjVic', 'TjRep', 'TjNR', 'ImE',
'ImSrc']
    \end{Verbatim}

    En la lista mostrada podemos observar los atributos que representan los
ámbitos que generalmente aparecen en nuestros acuerdos. Los
primeros 5 no pertenecen al grupo de atributos numéricos
descritos anteriormente, por lo que los descartaremos. Del resto de
atributos, he decidido escoger aquellos que me parecían más interesantes
a la hora de extraer conclusiones. Los atributos seleccionados son los
mostrados en el apartado de presentación de los datos.

A continuación, procedemos a representar los atributos seleccionados con
diferentes visualizaciones que nos permitan un correcto análisis y la
posterior extracción de conclusiones. Añadimos el código con el que realizamos este tipo de gráficas plot.bar, la segunda de las gráficas será la versión normalizada de la primera.

    \begin{tcolorbox}[breakable, size=fbox, boxrule=1pt, pad at break*=1mm,colback=cellbackground, colframe=cellborder]
\prompt{In}{incolor}{413}{\boxspacing}
\begin{Verbatim}[commandchars=\\\{\}]
\PY{n}{Ra} \PY{o}{=} \PY{n}{pd}\PY{o}{.}\PY{n}{crosstab}\PY{p}{(}\PY{n}{data}\PY{p}{[}\PY{l+s+s2}{\PYZdq{}}\PY{l+s+s2}{Reg}\PY{l+s+s2}{\PYZdq{}}\PY{p}{]}\PY{p}{,} \PY{n}{data}\PY{p}{[}\PY{l+s+s2}{\PYZdq{}}\PY{l+s+s2}{GRa}\PY{l+s+s2}{\PYZdq{}}\PY{p}{]}\PY{p}{)}
\PY{n}{Ra}\PY{o}{.}\PY{n}{plot}\PY{o}{.}\PY{n}{bar}\PY{p}{(}\PY{n}{stacked}\PY{o}{=}\PY{k+kc}{True}\PY{p}{,} \PY{n}{color}\PY{o}{=}\PY{p}{[}\PY{l+s+s2}{\PYZdq{}}\PY{l+s+s2}{lightgreen}\PY{l+s+s2}{\PYZdq{}}\PY{p}{,} \PY{l+s+s2}{\PYZdq{}}\PY{l+s+s2}{mediumseagreen}\PY{l+s+s2}{\PYZdq{}}\PY{p}{,} \PY{l+s+s2}{\PYZdq{}}\PY{l+s+s2}{seagreen}\PY{l+s+s2}{\PYZdq{}}\PY{p}{,} \PY{l+s+s2}{\PYZdq{}}\PY{l+s+s2}{darkgreen}\PY{l+s+s2}{\PYZdq{}}\PY{p}{]}\PY{p}{)}
\PY{n}{plt}\PY{o}{.}\PY{n}{title}\PY{p}{(}\PY{l+s+s2}{\PYZdq{}}\PY{l+s+s2}{Pactos con rasgos raciales por Región}\PY{l+s+s2}{\PYZdq{}}\PY{p}{)}
\PY{n}{plt}\PY{o}{.}\PY{n}{legend}\PY{p}{(}\PY{n}{title}\PY{o}{=}\PY{l+s+s2}{\PYZdq{}}\PY{l+s+s2}{mark}\PY{l+s+s2}{\PYZdq{}}\PY{p}{)}

\PY{n}{RaN} \PY{o}{=} \PY{n}{pd}\PY{o}{.}\PY{n}{crosstab}\PY{p}{(}\PY{n}{data}\PY{p}{[}\PY{l+s+s2}{\PYZdq{}}\PY{l+s+s2}{Reg}\PY{l+s+s2}{\PYZdq{}}\PY{p}{]}\PY{p}{,} \PY{n}{data}\PY{p}{[}\PY{l+s+s2}{\PYZdq{}}\PY{l+s+s2}{GRa}\PY{l+s+s2}{\PYZdq{}}\PY{p}{]}\PY{p}{,} \PY{n}{normalize}\PY{o}{=} \PY{l+s+s2}{\PYZdq{}}\PY{l+s+s2}{index}\PY{l+s+s2}{\PYZdq{}}\PY{p}{)}
\PY{n}{RaN}\PY{o}{.}\PY{n}{plot}\PY{o}{.}\PY{n}{bar}\PY{p}{(}\PY{n}{stacked}\PY{o}{=}\PY{k+kc}{True}\PY{p}{,} \PY{n}{color}\PY{o}{=}\PY{p}{[}\PY{l+s+s2}{\PYZdq{}}\PY{l+s+s2}{lightgreen}\PY{l+s+s2}{\PYZdq{}}\PY{p}{,} \PY{l+s+s2}{\PYZdq{}}\PY{l+s+s2}{mediumseagreen}\PY{l+s+s2}{\PYZdq{}}\PY{p}{,} \PY{l+s+s2}{\PYZdq{}}\PY{l+s+s2}{seagreen}\PY{l+s+s2}{\PYZdq{}}\PY{p}{,} \PY{l+s+s2}{\PYZdq{}}\PY{l+s+s2}{darkgreen}\PY{l+s+s2}{\PYZdq{}}\PY{p}{]}\PY{p}{)}
\PY{n}{plt}\PY{o}{.}\PY{n}{title}\PY{p}{(}\PY{l+s+s2}{\PYZdq{}}\PY{l+s+s2}{Pactos con rasgos raciales por Región (Normalizado)}\PY{l+s+s2}{\PYZdq{}}\PY{p}{)}
\PY{n}{plt}\PY{o}{.}\PY{n}{legend}\PY{p}{(}\PY{n}{title}\PY{o}{=}\PY{l+s+s2}{\PYZdq{}}\PY{l+s+s2}{mark}\PY{l+s+s2}{\PYZdq{}}\PY{p}{)}
\end{Verbatim}
\end{tcolorbox}
        
    \begin{center}
    \adjustimage{max size={0.8\linewidth}{0.35\paperheight}}{output_30_1.png}
    \end{center}
    { \hspace*{\fill} \\}
    
    \begin{center}
    \adjustimage{max size={0.5\linewidth}{0.35\paperheight}}{output_30_2.png}
    \end{center}
    { \hspace*{\fill} \\}
    
    En los histogramas podemos apreciar una clara diferencia entre el grado
de implicación racial de los tratados en África con respecto al resto de
zonas. Seguramente esto se deba a la mayor densidad de tribus, clanes y
grupos étnicos del continente africano en comparación al resto del
mundo. Por ejemplo, solo en Ghana, hay entre setenta y cien grupos
étnicos en una población estimada de unas 30.115.300 de personas.

Fuente: \url{https://es.wikipedia.org/wiki/Etnias\_de\_Ghana}

    \begin{center}
    \adjustimage{max size={0.5\linewidth}{0.35\paperheight}}{output_32_1.png}
    \end{center}
    { \hspace*{\fill} \\}
    
    \begin{center}
    \adjustimage{max size={0.5\linewidth}{0.35\paperheight}}{output_32_2.png}
    \end{center}
    { \hspace*{\fill} \\}
    
    Representando el aspecto religioso del acuerdo llegamos a un comportamiento muy similar al del diagrama de barras anterior. 
Comprobemos si existe algún tipo de relación entre las dos variables. Para ver de manera simple si las variables guardan relación haremos uso de la herramienta crosstab, cuyo código se muestra a continuación.

\begin{tcolorbox}[breakable, size=fbox, boxrule=1pt, pad at break*=1mm,colback=cellbackground, colframe=cellborder]
\prompt{In}{incolor}{402}{\boxspacing}
\begin{Verbatim}[commandchars=\\\{\}]
\PY{n}{RaRe} \PY{o}{=} \PY{n}{pd}\PY{o}{.}\PY{n}{crosstab}\PY{p}{(}\PY{n}{index}\PY{o}{=}\PY{n}{data}\PY{p}{[}\PY{l+s+s2}{\PYZdq{}}\PY{l+s+s2}{GRe}\PY{l+s+s2}{\PYZdq{}}\PY{p}{]}\PY{p}{,}
            \PY{n}{columns}\PY{o}{=}\PY{n}{data}\PY{p}{[}\PY{l+s+s2}{\PYZdq{}}\PY{l+s+s2}{GRa}\PY{l+s+s2}{\PYZdq{}}\PY{p}{]}\PY{p}{)}\PY{o}{.}\PY{n}{apply}\PY{p}{(}\PY{k}{lambda} \PY{n}{r}\PY{p}{:} \PY{n}{r}\PY{o}{/}\PY{n}{r}\PY{o}{.}\PY{n}{sum}\PY{p}{(}\PY{p}{)} \PY{o}{*}\PY{l+m+mi}{100}\PY{p}{,}
                                       \PY{n}{axis}\PY{o}{=}\PY{l+m+mi}{1}\PY{p}{)}\PY{o}{.}\PY{n}{plot}\PY{p}{(}\PY{n}{kind}\PY{o}{=}\PY{l+s+s2}{\PYZdq{}}\PY{l+s+s2}{bar}\PY{l+s+s2}{\PYZdq{}}\PY{p}{,} \PY{n}{color}\PY{o}{=}\PY{p}{[}\PY{l+s+s2}{\PYZdq{}}\PY{l+s+s2}{lightgreen}\PY{l+s+s2}{\PYZdq{}}\PY{p}{,} \PY{l+s+s2}{\PYZdq{}}\PY{l+s+s2}{mediumseagreen}\PY{l+s+s2}{\PYZdq{}}\PY{p}{,} \PY{l+s+s2}{\PYZdq{}}\PY{l+s+s2}{seagreen}\PY{l+s+s2}{\PYZdq{}}\PY{p}{,} \PY{l+s+s2}{\PYZdq{}}\PY{l+s+s2}{darkgreen}\PY{l+s+s2}{\PYZdq{}}\PY{p}{]}\PY{p}{)}

\PY{n}{plt}\PY{o}{.}\PY{n}{title}\PY{p}{(}\PY{l+s+s2}{\PYZdq{}}\PY{l+s+s2}{Relación entre la raza y la religión en los acuerdos}\PY{l+s+s2}{\PYZdq{}}\PY{p}{)}
\end{Verbatim}
\end{tcolorbox}
        
    \begin{center}
    \adjustimage{max size={0.5\linewidth}{0.35\paperheight}}{output_34_1.png}
    \end{center}
    { \hspace*{\fill} \\}
    
    La gráfica muestra de forma evidente un alto porcentaje de casos en los
que acuerdos de índole racial tienen que ver también con acuerdos con
medidas religiosas. Y del mismo modo, en los acuerdos sin menciones
raciales tampoco suelen aparecer implementaciones religiosas.

Al parecer, tenemos una gran
correlación entre la raza y la religión de las poblaciones indígenas, siendo la primera la que explica
en muchas ocasiones la existencia de la segunda y viceversa.
        
    \begin{center}
    \adjustimage{max size={0.5\linewidth}{0.35\paperheight}}{output_36_1.png}
    \end{center}
    { \hspace*{\fill} \\}
    
    \begin{center}
    \adjustimage{max size={0.5\linewidth}{0.35\paperheight}}{output_36_2.png}
    \end{center}
    { \hspace*{\fill} \\}
    
    Con las menciones a refugiados volvemos a encontrar un comportamiento similar. Veamos que ocurre al representar tanto el grado de raza como el grado de refugiados en una misma gráfica.
        
    \begin{center}
    \adjustimage{max size={0.5\linewidth}{0.35\paperheight}}{output_38_1.png}
    \end{center}
    { \hspace*{\fill} \\}
    
    En la representación gráfica anterior se intuye una tendencia en los pactos. Cuanto más se tiene en cuenta a los refugiados en el pacto, mayor es la presencia de referencias raciales en el mismo. La explicación obvia de este suceso es que los refugiados son generalmente de razas diferentes a la raza del país de origen o, que su raza no es bien recibida en un territorio determinado.
    
    \begin{center}
    \adjustimage{max size={0.5\linewidth}{0.35\paperheight}}{output_40_1.png}
    \end{center}
    { \hspace*{\fill} \\}
    
    Siguiendo la línea del análisis, volvemos a centrarnos en el aspecto racial de los pactos. En este caso hemos representado el grado de referencia a víctimas del conflicto y si existen medidas de carácter racial en el mismo. En la gráfica se aprecia un claro aumento en la aparición de acuerdos raciales conforme se incrementa el nivel de importancia que las víctimas del conflicto tienen en el pacto. Esta gráfica, junto con las anteriores, parece mostrar $\textit{a priori}$ la existencia de tendencias racistas que dan lugar a conflictos de odio entre los países y sus poblaciones indígenas.
    
    \begin{center}
    \adjustimage{max size={0.5\linewidth}{0.35\paperheight}}{output_42_1.png}
    \end{center}
    { \hspace*{\fill} \\}
    
    \begin{center}
    \adjustimage{max size={0.5\linewidth}{0.35\paperheight}}{output_42_2.png}
    \end{center}
    { \hspace*{\fill} \\}
    
    Por último, tenemos los derechos humanos, los cuales aparecen en un elevado número de acuerdos con poblaciones indígenas originales de África. En Ámerica, Asia y el Pacífico también tenemos un alto porcentaje de pactos que contienen medidas relacionadas con los derechos humanos.
    
    Un mayor número de representaciones han sido realizadas a la hora de elegir cuáles integrarían el documento final. Estas nos han ayudado a analizar mejor las tendencias y valores de nuestra base de datos. No han sido incluidas para evitar una extensión demasiado larga del informe.
    
 \subsection{Procedimiento y
herramientas}

    El procedimiento seguido y las herramientas utilizadas han sido expuestos y explicados a lo largo del desarrollo del informe. El documento escrito ha sido elaborado en Overleaf, un editor de documentos látex online. La parte de programación y representación de visualizaciones se ha realizado en Jupyter, con lenguaje Python. Las librerías empleadas han sido matplotlib y seaborn. Otra posibilidad habría sido realizar las visualizaciones en R, programa que hemos usado en otras asignaturas del máster.

\subsection{Dashboard}

En las páginas siguientes mostramos un exposición conjunta de las visualizaciones mostradas a lo largo de este apartado. 

\pagebreak

\section{PEC3: Informe de proyecto}\label{PEC3}

\subsection{Resumen}

El presente apartado introduce el proyecto de visualización de datos que se va ha presentar en la PEC4 con el conocimiento de los datos adquirido en la PEC2. La base de datos recopila una selección de pactos de paz que involucran poblaciones indígenas. El objetivo es describir las representaciones gráficas y el diseño escogidos para  visualizar las características y curiosidades de la base de datos.

\subsection{Introducción}

 El proyecto gira entorno a la selección de tratados de paz extraídos del proyecto $\textit{Peace Agreements}$ \cite{peace} de la Universidad de Edimburgo. El proyecto recoge un gran número de acuerdos de paz de diversa índole, firmados desde 1990 hasta la actualidad. En nuestro caso, abarcaremos únicamente aquellos tratados en los que participen o se tengan en cuenta a comunidades de poblaciones indígenas. 

La elección de las poblaciones indígenas como tema central guarda relación con mi interés por la diversidad de la especie humana. Me parece apasionante como cada una de estas poblaciones representa una cultura diferente y una forma completamente distinta de ver el mundo. Considero vital la preservación de las pocas comunidades de poblaciones indígenas que quedan hoy día y, por ese motivo, he decidido escoger estas como punto central del proyecto. En la $\textit{UNESCO}$ \cite{unesco} podemos aprender más acerca de estas poblaciones indígenas y su situación política y social actual.

El propósito del proyecto será, por lo tanto, elaborar una visualización que ayude a llamar la atención sobre la cuestión de las poblaciones indígenas, muchas veces olvidada. Con este objetivo en mente, trataremos de extraer conclusiones y curiosidades de la base de datos que puedan 
ser llamativas para el usuario. De este modo, buscaremos concienciar al usuario de la importancia de preservar la cultura, religión, forma de vida... de estas minorías. Además, intentaremos hacer comprender al usuario las difíciles condiciones en las que se encuentran y el peligro que corren si no las protegemos. 

La principal expectativa del proyecto queda perfectamente reflejada en el propósito establecido: Ser capaces de exponer de manera atractiva y clara la información extraída en la PEC2, de forma que los datos queden en manos del usuario y sea este el encargado de sacar sus propias conclusiones y opinión acerca de las poblaciones indígenas.

Para conseguir el resultado deseado, en este apartado llevaremos a cabo diversos procedimientos que nos ayudarán a crear la visualización final.

Primero comenzaremos describiendo el dataset, su estructura, el tipo de datos que recoge, las distintas categorías de datos existentes... A continuación, describiremos el proceso de trabajo al que someteremos a los datos para su correcta preparación. Tras esto, analizaremos las evidencias y conocimientos extraídos en la PEC2, recopilando la información que nos interesa mostrar en la visualización y el tipo de relaciones que queremos establecer entre los datos del dataset. Posteriormente, iniciaremos el apartado de descripción de las gráficas escogidas para representar nuestros datos y explicaremos el por qué de su elección. Además, añadiremos la descripción de las herramientas de interacción con el usuario, así como los filtros presentes en la visualización, el grado de importancia de cada gráfico y el carácter global de la visualización.

Tras este bloque de datos, desarrollaremos el apartado relacionado con el diseño de la visualización. Los principales puntos a tratar serán la gama de colores escogida, la tipografía que será utilizada, la estructura general de la visualización, los bloques en los que la dividiremos y el formato de la misma. Todo este bloque de diseño irá acompañado por un $\textit{Mockup}$ en el que dibujaremos con gran detalle cómo imaginamos la salida gráfica final del proyecto. Por último, haremos referencia a las visualizaciones que nos han inspirado en el proyecto.

\subsection{Descripción del dataset \label{dim}}

Este apartado ha sido tratado anteriormente en el apartado respectivo a la PEC2, "2.2 Presentación de los datos". Vuelve a ser incluido pues también formó parte de la PEC3.

El dataset recopila un total de 116 acuerdos de paz con diferentes medidas para la convivencia y preservación de comunidades de poblaciones indígenas. Para cada uno de los acuerdos disponemos de 265 metadatos o atributos, generando un total de 30740 registros. 

El enorme número de atributos dificulta la descripción e identificación de cada uno de ellos de manera individual. Por esta razón, hemos procedido a agrupar los atributos en grupos, según el tipo de datos que los constituyen. La descripción de la totalidad de los atributos puede ser encontrada en el manual de uso \cite{manual} de la base de datos proporcionado por $\textit{Peace Agreements}$. La clasificación de los atributos queda reflejada en los siguientes apartados.

\subsubsection{Atributos categóricos}

Los atributos categóricos nos ayudarán a identificar el acuerdo y definir sus características generales. Los atributos categóricos más relevantes con los que trabajaremos en el proyecto son los siguientes:

\begin{itemize}
    \item Con: Países/regiones/poblaciones que han llevado a cabo el acuerdo.
    
    \item Reg: Región del mundo en la que el conflicto resuelto por el acuerdo ha tenido lugar. Tenemos las siguientes:
   \begin{itemize}
   \item Africa (excl. MENA)
   
   \item Americas

   \item Asia and Pacific

   \item Europe and Eurasia

   \item Middle East and North Africa

   \item Cross-regional

   \item Other
   \end{itemize}
   
   \item Dat: Fecha en la que el acuerdo fue firmado o negociado. Formato: AAAA-MM-DD

    \item LOC1ISO: Localización del conflicto entre estados. (Asignada aleatoriamente) 
    
    \item LOC2ISO: Localización del conflicto entre estados. (Asignada aleatoriamente) 

    \item Contp: tipo de conflicto que resuelve el acuerdo. Tenemos los siguientes tipos:
     \begin{itemize}
   \item Government: Político o ideológico

   \item Territory: Territorial

   \item Government/territory: Político/territorial

   \item Inter-group: Conflictos entre grupos no estatales; los acuerdos pertinentes a menudo tratan con gobiernos revolucionarios provisionales.

   \item Other: Otros tipos de acuerdo
   \end{itemize}

    \item Agtp: Naturaleza primaria del acuerdo y el conflicto. Tenemos las siguientes naturalezas de acuerdo:
    \begin{itemize}
   \item Inter: Internacional

   \item Intra: Intranacional

   \item IntraLocal: Acuerdos relacionados con un conflicto intraestatal, pero con el objetivo de resolver problemas locales.

   \end{itemize}
   
    \item Status: Estado del acuerdo. Variable categórica con cuatro tipos de acuerdo disintos:
     \begin{itemize}
   \item Multiparty signed/agreed: Acuerdo firmado o claramente aceptado por más de un grupo de participantes opuestos (pero no necesariamente todos los grupos opuestos).
   
   \item Unilateral agreement: Acuerdo unilateral producido por una parte pero en respuesta a un acuerdo con la otra parte.

   \item Status unclear: Si no está claro qué parte del acuerdo se firmó o quién acordó, pero hay alguna indicación de la documentación circundante de que fue firmado / acordado.
   
   \item Agreement with Subsequent Status: Acuerdo propuesto basado en le diálogo entre las partes y una 'suposición' de lo que acordarán que, aunque no fue aceptado por las partes como un acuerdo, se convirtió la base para desarrollos posteriores.
   \end{itemize}
   
    \item Stage: Fase en el que se encuentra el acuerdo. Variable categórica con siete fases:
    \begin{itemize}
   \item Pre: Proceso de prenegociación.
   
   \item SubPar: Acuerdo parcial en algunos aspectos del acuerdo. Algunas partes del acuerdo siguen pendientes de discusión.

   \item SubComp: Similar a la etapa SubPar pero existe un intento real de resolver el conflicto.
   
   \item Imp: Implementación o renegociación con el objetivo de implementar un acuerdo anterior.
   
   \item Ren: Renovación con acuerdos cortos que buscan renovar los compromisos anteriores.
   
   \item Cea: Fase de alto el fuego o relacionadas. Acuerdos que comprometen total o parcialmente al alto el fuego o similares.

    \item Other: Otros tipos de acuerdos que se encuentran en estados distintos a los anteriores.
   \end{itemize}
\end{itemize}

\subsubsection{Atributos numéricos \label{num}}

Los atributos numéricos generalmente indicarán el grado de relación del acuerdo de paz con un ámbito concreto como, por ejemplo, si el acuerdo incluye medidas que tengan en cuenta distintas razas o si existe un factor religioso en el mismo. También tendremos una serie de atributos binarios que tomarán los valores 1 y 0, en función de si el acuerdo de paz aborda el tema especificado o no, respectivamente.

Para evitar extendernos a atributos que no tengan gran peso en nuestra selección de acuerdos indígenas, nos centraremos en aquellos que aparecen un mayor número de veces. 

El primer grupo de atributos numéricos, los que indican el grado de relación, irán expresados con un número del 0 al 3. Si el acuerdo no contiene referencias al ámbito especificado o similares, el valor se establece en 0. Si existen tales
referencias, la variable toma los valores de la siguiente manera:

\begin{itemize}
\tightlist
\item
  1: Solo se menciona el ámbito sin especificar detalles y este no tiene
  relevancia real en el acuerdo.
\item
  2: El acuerdo contiene medidas que hacen referencia al ámbito
  especificado o similares, y estas brindan más detalles sobre el modo
  de implementación de las medidas.
\item
  3: El acuerdo dispone de medidas sustantivas respecto
  al ámbito especificado, dando detalles e indicando compromiso con su
  implementación.
\end{itemize}
    Los atributos escogidos de este tipo son los siguientes:

\begin{itemize}
\item
  GRa: Razas/etnias/grupos nacionalistas. Señala las menciones en el
  acuerdo de raza, grupos étnicos, minorías nacionales, clanes u
  organizaciones 'tribales' similares.
\item
  GRe: Grupos religiosos. Hace referencia a cualquier mención a grupos
  religiosos, ya sea en términos de esos grupos o en términos de la
  inclusión de religiones. ~
\item
  GRef: Refugiados. Hace referencia a los refugiados y a las personas
  desplazadas (incluida la repatriación) que aparecen en el acuerdo de
  paz.
\item
  TjVic: Justicia transicional. Incluye cualquier disposición que brinde
  asistencia específica a las víctimas del conflicto o medidas que guarden alguna relación con aquellos que han sufrido el conflicto, incluidas víctimas desaparecidas
\end{itemize}

El resto de variables numéricas serán binarias, las cuales tomarán
el valor de 1 si el acuerdo de paz aborda el tema especificado en el
atributo. Si en cambio el tema no se aborda en absoluto en el acuerdo
de paz, el valor en esta variable es 0. Únicamente escogeremos un atributo de este tipo:

\begin{itemize}
\item
  HrGen: Derechos humanos/Estado de derecho. Variable binaria, toma el
  valor 1 si el acuerdo de paz incluye referencias a los derechos
  humanos o al derecho internacional. Si no aparecen tales disposiciones
  en el acuerdo, el valor de esta variable es 0.
\end{itemize}

\subsection{Proceso de trabajo}

El proceso de trabajo recoge todas aquellas actividades relacionadas con la limpieza, categorización y análisis del dataset, entre otras. En nuestro caso, la mayor parte estas tareas fueron realizadas en la PEC2 anterior. Los procedimientos llevados a cabo fueron los siguientes:

\begin{itemize}
    \item Evaluación inicial de los datos. Para ello, mostramos las cinco primeras líneas de nuestro dataset, correspondientes a cinco acuerdos de paz donde se tienen en cuenta las poblaciones indígenas. De este modo, evaluamos que la base de datos no presenta errores evidentes y comenzamos a familiarizarnos con la misma.
    
    \item Extracción de las dimensiones del dataset. Obtenemos el número de filas (acuerdos) y el número de columnas (atributos), tal y como vemos en el apartado \ref{dim}.
    
    \item Limpieza de datos vacíos o nulos. Primero comprobamos si aparecen datos vacíos o nulos en el dataset. Al no ser el caso, no se toman medidas al respecto.
    
    \item Análisis preliminar de los datos. Representamos los atributos categóricos en histogramas para conocer mejor las características generales del conjunto de acuerdos con el que estamos trabajando. Extraemos las primeras conclusiones sobre el tipo de acuerdos que generalmente forman nuestra selección.
    
    \item Selección de atributos. La enorme variedad de atributos, nos lleva a centrarnos en un grupo reducido de ellos, aquellos que más puedan interesarnos a la hora de analizar los acuerdos con poblaciones indígenas. Los atributos numéricos presentados en el apartado \ref{num} son escogidos según la cantidad de veces e importancia con la que aparecen en los pactos de paz.
    
    \item Creación de nuevos de atributos. Para conseguir la distribución de los pactos a lo largo de los años, añadiremos una columna ('Year') que indique únicamente el valor del año en el que se produjo el acuerdo. 
    
    Para un correcta representación por países de los pactos, podría ser conveniente modificar la columna 'Con'. En algunas ocasiones
    aparecen tanto el país como la población indígena implicada, dificultando recopilar todos los acuerdos de un mismo país. Por ejemplo, para la India tenemos los siguientes valores: 'India-Bodoland', 'India-Darjeeling' o 'India-Tripura', entre otros. Esto imposibilita agrupar todos los acuerdos que involucran a la India en un atributo de recuento de tratados.
    
    \item Limpieza del resto de atributos. El resto de atributos, pese a que no se descarta su utilidad en el análisis, serán obviados en el proyecto, permitiéndonos reducir enormemente la base de datos.
    
    \item Categorización de los datos. Si bien podríamos modificar los nombres de las variables numéricas, por ejemplo, indicando con la etiqueta 'Muy implicado' aquellos acuerdos que presenten un tres en alguna de los ámbitos, la enumeración (del 0 al 3) de los rangos es bastante intuitiva y se dejará como tal. También podríamos proceder de forma parecida con los atributos binarios, pero mantendremos la nomenclatura original.
\end{itemize}

\subsection{Evidencias y conocimientos obtenidos en la PEC2} \label{ev}

La exploración de los datos llevada a cabo en la PEC2 nos permitió extraer una serie de evidencias y conocimientos que emplearemos en la elaboración de las visualizaciones del proyecto. Las representaciones gráficas fueron creadas en Jupyter con lenguaje Python. Se uso principalmente la función $\textit{crosstab}$ de la librería pandas \cite{pandas}.

A partir de estos gráficos de barras pudimos conocer como se distribuyen las clases de acuerdos por regiones y establecer una serie de causas que explicarán las tendencias y $\textit{outliers}$ destacados. Las conclusiones más relevantes extraídas de la PEC2 fueron las siguientes:

\begin{itemize}
    \item Existe un mayor grado de implicación racial en los tratados en África con respecto al resto de zonas. Seguramente esto se deba a la mayor densidad de tribus, clanes y grupos étnicos del continente africano en comparación al resto del mundo. Por ejemplo, solo en Ghana, hay entre setenta y cien grupos étnicos en una población estimada de unas 30.115.300 de personas \cite{ghana}.
    
    \item Los acuerdos con medidas religiosas mostraban una distribución similar a los acuerdos anteriores. Decidimos representar la relación entre ambos, apareciendo un alto porcentaje de casos en los que acuerdos de índole racial tenían que ver también con acuerdos de índole religiosa. Al parecer, tenemos una gran correlación entre la raza y la religión de las poblaciones indígenas, siendo la primera la que explica en muchas ocasiones la existencia de la segunda y viceversa.
    
    \item A partir de este punto, decidimos centrar el proyecto en el aspecto racial de los acuerdos con indígenas. Volvemos a repetir el proceso anterior, pero en este caso con el grado de implicación del pacto con los refugiados. De aquí extraemos que, cuanto más se tiene en cuenta a los refugiados en el pacto, mayor es la presencia de referencias raciales en el mismo. La explicación obvia de este suceso es que los refugiados son generalmente de razas diferentes a la raza del país de origen o, que su raza no es bien recibida en un territorio determinado.
    
    \item Comprobamos si existen un mayor número de referencias a víctimas en los tratados de índole racial. Los resultados reflejan un claro aumento en la aparición de acuerdos raciales conforme se incrementa el nivel de importancia que las víctimas del conflicto tienen en el pacto. Esta gráfica, junto con las anteriores, parece mostrar a priori la existencia de tendencias racistas que dan lugar a conflictos de odio entre los países y sus poblaciones indígenas.
    
    \item Por último, representamos los derechos humanos por regiones, los cuales aparecen en un elevado número de acuerdos con poblaciones indígenas originales de África. En América, Asia y el Pacífico también tenemos un alto porcentaje de pactos que contienen medidas relacionadas con los derechos humanos.
    
\end{itemize}

Por supuesto, las evidencias y conocimientos enumerados serán usados en el proyecto, permitiéndonos saber en qué ámbitos deberemos enfocarnos para crear visualizaciones que contengan la mayor cantidad de información sobre el tipo de pactos de paz que estamos abordando.

\subsection{Tipos de relaciones}

A parte de una representación individual detallada para cada uno de los acuerdos de paz, el objetivo fundamental del proyecto de visualización será representar las relaciones existentes entre los pactos de cada una de las regiones. En un principio buscaremos patrones que se repitan dentro de los pactos de una sola región, intentando ver si existe una correspondencia entre el tipo de acuerdos y la región donde tienen lugar. En la PEC2 podemos observar varios ejemplos; en muchas de las evidencias extraídas (sección \ref{ev}) aparecen múltiples tendencias en los acuerdos, sobre todo en los que tienen lugar en África.

Posteriormente, se representarán de manera conjunta las distintas regiones en función de los atributos señalados en la sección \ref{num}. De este modo, se podrán comparar los resultados en busca de tendencias remarcables entre los tratados correspondientes a las diferentes regiones.

Por otro lado, en la visualización trataremos de representar la distribución que siguen los acuerdos cuando comparamos dos atributos al mismo tiempo. Podemos apreciar varios ejemplos de esto en el apartado \ref{ev}. Uno de ellos consistiría en representar el grado racial frente al grado religioso de los acuerdos, apareciendo cierta relación y proporcionalidad entre ambos atributos.

Aunque no será vital para alcanzar los objetivos establecidos, también se representarán los atributos categóricos para las distintas regiones, facilitando la clasificación del tipo de tratados que predomina en la selección de pactos.

En resumen, el proyecto tendrá dos objetivos principales con respecto a los datos a exponer:
\begin{enumerate}
    \item Identificar las características generales que aparezcan en los acuerdos de unas regiones u otras.
    \item Identificar posibles tendencias y relaciones entre los tipos acuerdos según su implicación en los distintos ámbitos (raza, religión, víctimas...)
\end{enumerate}

\subsection{Selección de gráficas y descripción}\label{graf}

Dado que buscamos mostrar diversos factores en un misma visualización, tendremos que recurrir a múltiples tipos de gráficas y herramientas a la hora de representar cada uno de ellos. Según el tipo de hecho que queramos representar tendremos los siguientes tipos de gráficas: 

\begin{itemize}
    \item La exposición de los 116 pactos de paz que integran nuestro dataset se hará de manera conjunta en una línea temporal. De este modo, los acuerdos se colocarán en orden cronológico según la fecha en la que fueron firmados. La corta extensión de la base de datos nos permite agrupar todos los pactos y ordenarlos de modo que el usuario pueda ir seleccionando y aprendiendo sobre las características generales de cada uno. La interacción del usuario con los datos se detallará en mayor medida en el siguiente apartado (sección \ref{gob}). La línea temporal irá acompañada por un histograma que muestre la distribución del número de pactos a lo largo de los años. De este modo, podremos apreciar donde hubo un mayor cúmulo de los mismos, indicando etapas conflictivas o sucesos históricos relevantes. También podrá venir acompañado de un mapa geográfico que nos indique de manera visual dónde se produjo el acuerdo. Los elementos descritos aparecen representados en la primera diapositiva del $\textit{Mockup}$.
    
    \item La visualización de las características generales de una región concreta se realizará mediante gráficos de barras en los cuales se representarán los distintos atributos (racial, religioso, refugiados...). Esto nos permitirá ver el grado de implicación de los acuerdos de una región concreta en un ámbito determinado. Los elementos descritos aparecen representados en la segunda diapositiva del $\textit{Mockup}$.
    
    \item La comparación de las características de varias regiones se hará combinando en una única representación las gráficas de barras anteriores. De este modo, podremos ver en qué regiones aparecen más un tipo de pactos que en otras. Añadir un mapa geográfico donde se señale el grado de implicación de los tratados en un ámbito concreto con colores que varíen en intensidad también podría ayudar al usuario. Observando el mapa podríamos saber de manera bastante intuitiva donde se concentra la mayoría de pactos de un tipo u otro. Los elementos descritos aparecen representados en la tercera y cuarta diapositiva del $\textit{Mockup}$.
    
    \item Las relaciones entre los atributos se realizarán de manera conjunta para todas las regiones, pues no buscamos analizar si hay un comportamiento concreto en una región, sino si la tendencia puede aplicarse a la mayoría de acuerdos de paz. En este caso, la utilización de un gráfico de barras que represente uno de los atributos en función del otro, por ejemplo raza frente a religión, nos permitirá una buena comprensión de la relación entre ambos. En este apartado también podría ayudarnos una tabla de contingencia que nos señale si los grados de un atributo se corresponde con los del otro. Es decir, si, por ejemplo, la mayor parte de pactos con rango racial 2 tienen a su vez rango religioso 2. De esta forma, podremos deducir de manera algo más estadística si los atributos guardan relación. Los elementos descritos aparecen representados en la quinta diapositiva del $\textit{Mockup}$.
    
    \item Por último, los atributos categóricos vendrán representados por gráficas de columnas que nos permitan ver las características generales de la selección de tratados con la que estamos trabajando (tipo de conflicto, naturaleza, estado del acuerdo...). Estas gráficas se realizarán para el conjunto de todas las regiones. Los elementos descritos aparecen representados en la sexta diapositiva del $\textit{Mockup}$.
\end{itemize}

\subsection{Descripción de la gobernanza de datos}\label{gob}

El conjunto de pactos de paz seleccionados permite una gran variedad de enfoques distintos según qué información queramos extraer. Es por ello que una parte vital de nuestra visualización será el uso de la interacción, permitiendo que el usuario sea el encargado de crear sus propias representaciones. De este modo, cada usuario podrá abordar el análisis de la base de datos de la forma que considere más adecuada.

El principal mecanismo de interacción vendrá ligado a la línea temporal donde se presentan los 116 acuerdos de paz en orden cronológico. El usuario será capaz de moverse en el tiempo a través de los distintos pactos. Para cada pacto aparecerá una pestaña emergente con los datos básicos del mismo: nombre del acuerdo, fecha, países y regiones implicados, tipo de acuerdo, estado de la negociación, descripción del acuerdo.... A partir de la información extraída de esta línea temporal el usuario podrá ir haciéndose una idea sobre qué contiene cada uno de los pactos y podrá centrarse en un único pacto si así lo desea.

El usuario podrá configurar el tipo de acuerdos que aparecen en la línea temporal modificando una serie de filtros que le permitirán especificar características como: región, naturaleza del conflicto, año del acuerdo...

Otra opción alternativa o complementaria es la de un mapa geográfico interactivo en el que aparezcan marcados los puntos donde tuvieron lugar los acuerdos los cuales, por lo general, indicarán la existencia de población indígena en la zona. En algunos casos habrá más de un tratado por punto, por lo que un desplegable mostrará la lista de acuerdos desarrollados en dicho punto.

El mapa también contará con una serie de filtros habilitando representaciones que reflejen distintas versiones de la distribución de los acuerdos alrededor del mundo. Los filtros presentarán los atributos destacados en la sección \ref{dim}. Los atributos categóricos presentarán un color distinto para cada una de las categorías posibles, mientras que los atributos numéricos mostrarán en un mismo color pero con intensidades variables según el grado de implicación.

Las gráficas de barras que se emplearán para exponer las características generales de cada región irán incluidas en un gráfico secundario. El usuario podrá elegir libremente la región y característica que desea representar. 

Del mismo modo, las gráficas de barras utilizadas para la comparación de las distintas regiones, aparecerán en una gráfica secundaria donde el usuario podrá escoger qué atributo comparar entre las distintas regiones. El usuario podrá elegir el tipo de representación que desee: frecuencia, proporción o una representación geográfica que indique el grado de implicación de cada zona.

Por último, tendremos las gráficas de barras donde representaremos las relaciones entre los atributos. Estás se localizarán en una nueva gráfica secundaria en la que el usuario podrá escoger que atributos desea comparar. 

El conjunto de las visualizaciones y de los gráficos, primarios y secundarios, dará lugar a una historia global sobre la que el usuario podrá ir desplazándose desde lo individual a lo global. De esta forma, cada usuario podrá acceder a la información proporcionada en la visualización de manera adaptada a sus intereses. Si desea conocer un dato específico de un pacto determinado, podrá hacerlo en la línea temporal si, en cambio, desea ver cómo son generalmente los acuerdos que involucran poblaciones indígenas, recurrirá a las visualizaciones globales.

\subsection{Diseño}\label{diseno}

El diseño de la visualización queda expuesto en el $\textit{Mockup}$ adjunto (sección \ref{mo}). Las principales características de nuestro diseño son las siguientes:
\begin{itemize}
    \item La gama de colores se centrará principalmente en una paleta de verdes. En las gráficas, los verdes más intensos indicarán un mayor grado de implicación mientras que los verdes más suaves, indicarán un bajo nivel de implicación.
    
    \item La tipografía escogida para el $\textit{Mockup}$ es 'Calibri (Body)'. Esta tipografía no es definitiva y podría variar en la visualización final.
    
    \item La estructura de la visualización sigue el orden de las gráficas indicadas en la sección \ref{graf}. La visualización comienza con representaciones específicas, individuales para cada tratado. Conforme avanzamos procedemos a un análisis más colectivo con gráficas de comparación y relación.
    
    \item La visualización quedará dividida en dos bloques principales: una primera parte individual orientada a los detalles y una segunda parte colectiva orientada a la comparación.
    
    \item El formato de la visualización será digital por el momento. Para su elaboración, haremos uso de la aplicación $\textit{Tableau}$ \cite{tableau}.
\end{itemize}
 
\subsection{Mockup}\label{mo}

\pagebreak

\subsection{Inspiración}\label{inspi}

La inspiración para crear la línea temporal que recoge los pactos de paz ha sido extraída de la página web  $\textit{information is beautiful}$ \cite{beau}, de la visualización titulada 'Based on a True True Story?' \cite{true}. La visualización muestra una serie de películas basadas en hechos reales divididas por escenas. De cada escena se indica su nivel de veracidad y se señala qué ocurrió realmente y qué no.

\begin{figure}[h]
\includegraphics[scale=2]{Inspi1.png}
\caption{Ejemplo de la visualización en la que nos hemos inspirado. La visualización muestra la veracidad de las escenas de la película 'Bohemian Rhapsody'.}
\end{figure}

El resto de ideas se han basado en los resultados extraídos de la PEC2 y en algunas visualizaciones vistas en la página de ejemplos de $\textit{Tableau}$ \cite{tab}

\section{PEC4: Creación de la visualización} \label{PEC4}

\subsection{Proyecto final}

El proyecto de visualización final presentará diversas variaciones respecto a las propuestas y el Mockup \ref{mo}
desarrollados en la PEC3. En algunos casos ni siquiera se conservarán todas las conclusiones extraídas en el análisis de la PEC2. Una serie de nuevas evidencias obtenidas conforme realizábamos el proceso de creación de la visualización, y los inconvenientes propios de una herramienta como Tableau en la que somos primerizos, nos han llevado a modificar el rumbo que habíamos establecido en las PECs anteriores. Los cambios y adiciones incorporadas en esta última parte serán comentadas y explicadas en detalle a lo largo de los siguientes apartados. También incluiremos aquellos procedimientos que  continúen con lo marcado en el diseño del proyecto.

Una vez finalicemos la descripción de la visualización, procederemos a analizar el resultado obtenido. Comprobaremos que objetivos planteados han sido cumplidos y cuáles no. De este modo, responderemos a la pregunta crucial que se le planteaba desde un inicio a nuestra visualización. ¿Responde al propósito para el que fue creada?

A continuación, añadiremos aquellos procedimientos técnicos que han habilitado la correcta implementación de los datos a la visualización. Presentaremos las limpiezas y modelados de datos que hemos realizado sobre la base de datos original para llegar al conjunto que utilizaremos en la visualización final.

Posteriormente, identificaremos el tipo de gráficos empleados en la visualización, incluyendo una explicación detallada de su elección.

Por último, recapacitaremos acerca del resultado obtenido, dando una valoración personal de la visualización y analizando los retos y el esfuerzo que ha supuesto la creación de la visualización. 

\subsubsection{Explicación del proyecto de visualización final}

El proyecto de visualización final constará de dos vistas ("Dashboard") con funciones distintas, pero que en conjunto generarán visualización global que abarca desde lo individual y específico, hasta lo grupal y genérico. 

En la primera vista, titulada "PACTOS DE PAZ CON PUEBLOS INDÍGENAS IMPLICADOS. LÍNEA TEMPORAL" se presentará en un eje cronológico los 126 pactos que incluye nuestro dataset. Desde 1990 hasta 2018 podremos observar cada pacto con su nombre y características generales, así como su localización geográfica en un mapa mundial y su grado de implicación en los atributos que destacábamos en las PECs 2 y 3. También se incorporará un distribución anual en la que se podrá ver la frecuencia de los pactos en busca de conflictos o situaciones que hayan propiciado un incremento o descenso en los mismos.

En la segunda vista, titulada "ELEMENTOS DE COMPARACIÓN POR PAÍSES/REGIONES" se incluirán los grados de implicación medios por país, es decir, cuál suele ser la implicación de los pactos llevados a cabo en dicho país respecto a los distintos atributos. De este modo, se podrán comparar los distintos países y el tipo de conflictos que predominan en las distintas zonas. También se incluye un apartado en el que se recogen el número de pactos y su grado de implicación para las 4 regiones que componen nuestra base de datos. Aquí podremos comparar de un modo más general que pactos caracterizan cada una de las regiones.

Ambas vistas incluirán diversos filtros que permitirán especificar las condiciones que el usuario desee limitar y estarán integradas en una única historia ("Story"), dejando al usuario al mando del tipo de visualización que busque analizar. En la siguiente imagen podemos ver como quedaría la historia que compone nuestra visualización.

\begin{figure}[h!]
    \caption{Historia: PACTOS DE PAZ CONTRA LA EXTINCIÓN DE LAS POBLACIONES INDÍGENAS. Visualización final}
\includegraphics[scale=2]{Vista3.png}
\end{figure}

\subsubsection{Consecución de objetivos}

Pese a algunas modificaciones con respecto al Mockup \ref{mo} incluido en la PEC3, la mayoría de objetivos y propuestas desarrolladas en el apartado de diseño han podido ser llevadas a cabo de manera acorde a la establecida. Una de las ideas fue la de introducir una línea temporal con los pactos ordenados cronológicamente, la cual ha podido ser perfectamente implementada en la visualización final. La línea temporal actúa como un "diccionario de pactos", y en ella podemos encontrar toda la información necesaria sobre un determinado acuerdo. Tanto el mapa geográfico como la distribución anual de los pactos han sido incluidas. Por tanto, damos por satisfactoria la consecución de esta propuesta, la cual albergaba un lugar fundamental y central en nuestro proyecto de diseño.

Otro punto importante en el diseño planteado era añadir gráficas que permitieran comparar las características de los tratados en las distintas regiones. Pese a que este apartado ha sido incluido, no goza de la relevancia que se le confería en la PEC3, donde presentaba varias formas de visualización, además de filtros, permitiendo una comprensión total de los pactos por región. En el proyecto final ha sido incorporada una sección en la que se pueden apreciar la cantidad de pactos por grado de implicación y región, sin llegar a permitir la extracción de conclusiones que buscábamos en el diseño conceptual.

Sin embargo, la ausencia de una sección centrada en las regiones no ha conllevado una pérdida de información en la visualización, sino todo lo contrario. En lugar de centrarnos en los pactos por regiones, hemos decidido considerar el estudio y comparación de los países, los cuales nos siguen arrojando cómo son los pactos en las regiones, pero de manera más específica y concreta. De este modo, hemos creado una segunda vista en la que se comparan las medias de los países para cada ámbito de implicación. Se puede considerar este elemento como una especie de "ranking" en el que podemos comparar cuánto se "involucra" cada país en sus pactos.

Por último, las gráficas con las características generales de los tratados (estado, tipo, etapa...) han sido sustituidas por filtros que nos permiten determinar que clase de pactos deseamos estudiar, pudiéndonos centrar en pactos pendientes de ser firmados, por tipo político, etc. El objetivo de esto es eliminar una sección que no aportaba gran información al usuario y combinarla con un apartado más general en el que se pueden extraer mejores conclusiones.

En definitiva, todos los cambios implementados han buscado mejorar el boceto inicial a partir de la información adquirida en el trabajo realizado en Tableau durante la PEC4.

\subsubsection{Descripción técnica del proyecto}

El apartado técnico del proyecto se ha centrado casi exclusivamente en el uso de Tableau, y las herramientas que pone a nuestra disposición para la elaboración de visualizaciones. De manera externa, han sido necesarias una serie de modificaciones en la base de datos realizadas a través de Python, en Jupyter. Esta sección de trabajo ha consistido en la extracción por separado de todos los países y la cantidad de veces que aparecía cada uno en los múltiples pactos. Mediante un documento .ipynb, incluido en GitHub, se ha logrado dicho objetivo. Para conseguirlo también ha sido necesario un archivo con los nombres en inglés de los países, con el que hemos comparado los nombres en la búsqueda. Junto con los nombres aparecen algunos de los datos que queremos que posteriormente aparezcan en la visualización. Las librerías empleadas han sido numpy \cite{python} y pandas \cite{pandas}.

A parte de los análisis preliminares realizados en Python durante la PEC2, un gran volumen del análisis ha sido llevado a cabo en Tableau, el cual nos ha habilitado una plataforma en la que representar de forma intuitiva y rápida diversas visualizaciones. Esta fase, como es de esperar, fue realizada de forma previa a la visualización final, y nos permitió familiarizarnos con el código y las posibilidades de Tableau. De aquí extrajimos algunos de los puntos que nos hicieron cambiar de opinión respecto al estudio de los países en lugar de las regiones. De este modo, creamos un repositorio de gráficas con las que profundizamos en los pactos y de las que seleccionamos las encargadas de representar nuestra base de datos en la visualización final.

\subsubsection{Descripción de los gráficos empleados}

Como ha sido explicado a lo largo de todo este apartado 4, la visualización en Tableau ha quedado de la siguiente manera:

\begin{itemize}
    \item Vista 1: "PACTOS DE PAZ CON PUEBLOS INDÍGENAS IMPLICADOS. LÍNEA TEMPORAL". Se compone de los siguientes elementos:
    \begin{itemize}
        \item Sección de filtrado: Consiste en cuatro filtros que permiten al usuario escoger la clase de conflictos que quiere estudiar en la vista. Los filtros son: Tipo de conflicto, Naturaleza del conflicto, Estado del conflicto y Fase del conflicto, cada uno con sus opciones respectivas. Seleccionando unas opciones u otras el usuario podrá alcanzar la representación de pactos deseada.
        
        \item Línea temporal: Constituye el elemento central de la vista, presentando los 126 pactos que recoge nuestro dataset en un eje cronológico. Moviendo el cursor por encima de los acuerdos se genera una pestaña emergente en la que se muestran las características generales del acuerdo (Año, país, nombre del pacto, estado del pacto...).
        
        \item Mapa geográfico: Consiste en un mapa mundial con los países que aparecen en los acuerdos. El país exacto en el que tiene lugar el acuerdo se señala en el mapa cuando situamos el cursor en la línea temporal correspondiente, a partir de una Acción (" Action") de tipo Destacar ("Highlight"). De este modo, podemos conocer en más detalle la región y localización del acuerdo.
        
        \item Barras de implicación: Algo que no se incluía en el Mockup pero que ha sido añadido en la versión final, es una ventana que indica el grado de implicación del proyecto seleccionado en la línea temporal. Mediante una Acción, en este caso de tipo Filtro ("Filter") conseguimos que únicamente aparezcan reflejados los grados de implicación para los atributos Racial, Religioso, Refugiados, Derechos Humanos y Víctimas para cada pacto.
        
        \item Distribución anual: Se trata de un histograma que abarca los años en los que ocurrieron los pactos (1990-2018), y muestra el número total de ellos que tuvieron lugar cada año. También interacciona con una Acción de tipo Filter con la línea temporal, mostrando la distribución del país que indica el tratado seleccionado. De aquí podremos extraer si el pacto tuvo lugar en un momento agitado de la historia del país o fue debido a otros sucesos. 
    \end{itemize}
    
\begin{figure}[h!]
    \caption{Vista 1: PACTOS DE PAZ CON PUEBLOS INDÍGENAS IMPLICADOS. LÍNEA TEMPORAL. Visualización final}
\includegraphics[scale=2]{Vista1.png}
\end{figure}

    \item Vista 2: "ELEMENTOS DE COMPARACIÓN POR PAÍSES/REGIONES". Se compone de los siguientes elementos:
    \begin{itemize}
        \item Sección de filtrado: Consiste en un filtro que permite al usuario escoger la región que quiere estudiar en la vista, África, América, Ásia y el Pacífico... Seleccionando unas opciones u otras, el usuario podrá alcanzar la representación de pactos deseada.
        
        \item Barras de implicación: De forma similar a las representaciones incluidas en el Mockup, se incluyen gráficos de barras que permiten ver los grados de implicación de los países con sus poblaciones indígenas. En esta medida se tiene en cuenta la media de los grados de implicación de todos lo pactos llevados a cabo en un mismo país, por ejemplo, los 18 pactos de Colombia serán promediados. Se han abordado los acuerdos por países pues considero que son más concretos y representan mejor la realidad que el cúmulo de todos los pactos de una misma región. Para cada país aparece indicado su código, nombre completo y región, así como las 5 medias de los atributos que estudiamos. También se ha añadido una línea de referencia que nos permite ver el grado de implicación total de cada país cuando tenemos en cuenta los 5 ámbitos. Podríamos comparar de este modo dónde se tiene más en cuenta a las poblaciones indígenas.
        
        \item Barras de implicación por regiones: En este caso las regiones aparecen en una única barra con niveles que indican el grado de implicación según la intensidad de su color (suave = 0, intenso = 3). Así, de una manera sencilla, conseguimos seguir incluyendo una comparativa regional de los tipos de acuerdo. Este elemento reaccionará mostrando únicamente el continente al que pertenece el país que pulsemos en las barras de implicación por países (Acción >> Filter).
        
        \begin{figure}[h]
\includegraphics[scale=2]{Vista2.png}
\caption{Vista 2: ELEMENTOS DE COMPARACIÓN POR PAÍSES/REGIONES. Visualización final}
\end{figure}
        
    \end{itemize}
\end{itemize}

\subsubsection{Esfuerzo y resultado. Retos}

La dedicación invertida en este proyecto de visualización ha sido muy alta. Contando el tiempo empleado para finalizar las dos PECs anteriores, absolutamente imprescindibles para llevar a buen puerto el proyecto, ha supuesto un gran esfuerzo y muchas horas de trabajo y análisis. Todas las etapas del proceso han sido importantes y han ido sumándose las unas a las otras para dar nacimiento a una visualización que, pese a ser la primera realizada en estas condiciones, cumple con creces las expectativas que me había impuesto al iniciar la asignatura. 

Los primeros pasos del proyecto, pese a ser familiares, pues ya habíamos emprendido procedimientos de limpieza y análisis de datos en otras asignaturas, requirieron un gran esfuerzo para asentar unas bases y conocimientos que nos permitieran generar una visualización explicativa de los datos. El principal reto fue tratar con una base de datos con semejante número de atributos, 265. La selección del conjunto final de atributos con el que trabajaríamos no fue sencilla, y fueron necesarios diversos enfoques para dar con el adecuado en la PEC2 y PEC3.

En la siguiente etapa tuvimos que desarrollar un proyecto de visualización desde 0, proponiendo un esquema de diseño que respondiera a las necesidades que queríamos responder a través de la base de datos. Pese a ser la primera vez que trabajaba en un proyecto de estas características, creo que la buena realización de la PEC2, junto con los conocimientos adquiridos en la PEC1 sobre buenas visualizaciones, me permitieron desarrollar un buen modelo conceptual. Gran parte de los elementos definidos en este apartado han podido ser implementados en la visualización final algo que, con mi escaso conocimiento previo de la herramienta Tableau, podría parecer difícil en un primer momento. Todo esto sirvió para crear un Mockup con el que ya se podía empezar a ver esbozos del resultado que buscábamos.

Finalmente, en esta última práctica hemos podido crear la visualización que teníamos en mente. Analizando el resultado final, considero que hemos elaborado una visualización que mezcla innovación y estética con utilidad y simpleza. La línea temporal, inspirada en una visualización sobre películas \cite{true}, \ref{inspi}, aporta una visión distinta de los acuerdos, dando una dimensión temporal muy difícil de alcanzar de otro modo. Los filtros facilitan la interacción con el usuario y permiten profundizar en aspectos muy específicos de los pactos. Las diagramas de barras aportan toda la información incluida en los tratados y capacitan al usuario con una forma sencilla y directa de comparación. El mapa aporta colorido y ayuda a localizar las poblaciones indígenas. También llama la atención la diversidad de interacciones que tienen lugar en la visualización, consiguiendo atraer al usuario a seguir indagando, justo lo que buscábamos conseguir. 

Con respecto a los elementos de diseño indicados en la PEC3 (\ref{diseno}), hemos logrado mantener la gama de colores verdes, consiguiendo una sensación de conjunto y armonía entre todas las ventanas que componen la visualización.  La tipografía ha sido modificada a "Tableau Bold", ya que permitía una mejor lectura y se asemejaba más llamativa al lector.

Implementar todos estos detalles ha requerido un proceso de adaptación intensivo a la herramienta Tableau. Inicialmente fue complicado entender en que consistía y como funcionaba pero, con la ayuda de la gran variedad de tutoriales en su web y una curva de aprendizaje bastante rápida, he sido capaz de alcanzar un nivel que me permite elaborar visualizaciones bastante avanzadas. En el futuro me planteo seguir usando esta herramienta para crear mis visualizaciones, pues la encuentro muy útil y facilita mucho la labor llevada a cabo.

Los datos con los que hemos operado a lo largo de todo el proyecto nos han ido mostrando distintas tendencias. En un inicio vimos como se relacionaban las implicaciones de acuerdo con los acuerdos de tipo racial, concluyendo que los pactos se volvían más religiosos y violentos conforme mayor era el factor racial del mismo. En la versión final de la visualización no se han incluido las gráficas que permitían extraer dicha información, pues fundamentalmente se llegaba a esos resultados por la influencia de los pactos de la región de África. Por tanto, se ha decidido aplicar un enfoque más centrado en la riqueza cultural de las poblaciones indígenas, que haga resaltar su valor al mundo. De este modo, se ha optado por mostrar los grados de implicación con tal de enseñar al usuario la diversidad de razas y religiones de estas culturas. También se han incluido las arduas condiciones a las que se ven sometidos a causa de estas diferencias con la población mayoritaria: obligados a huir refugiados, mayor referencia a víctimas e inclusión y modificación de derechos humanos.

En conclusión, la visualización nos aporta información a la que muy difícilmente hubiéramos podido acceder indagando en la base de datos. Nos resume los pactos de paz que incluyen poblaciones indígenas y nos permite conocer mejor cuales son las condiciones que generalmente constituyen estos acuerdos. Además, se realiza de manera que el usuario se sienta atraído a continuar leyendo y a seguir profundizando consiguiendo, quizás, que se implique en la conservación del tesoro cultural que integran las poblaciones indígenas.

\pagebreak

\section{Bibliografía}

\begin{thebibliography}{X}
\bibitem{peace}https://www.peaceagreements.org

\bibitem{unesco} https://es.unesco.org/indigenous-peoples

\bibitem{manual} https://www.peaceagreements.org/files/PA-X, visitado el 2/5/2020.

\bibitem{python} https://numpy.org/, visitado el 16/4/2020.

\bibitem{pandas} https://pandas.pydata.org/pandas-docs/stable/reference/api/pandas.crosstab.html, visitado el 16/4/2020.

\bibitem{ghana} https://es.wikipedia.org/wiki/Etnias-de-Ghana, visitado el 16/4/2020.

\bibitem{tableau} https://www.tableau.com/, visitado el 1/5/2020.

\bibitem{beau} https://informationisbeautiful.net/, visitado el 4/5/2020.

\bibitem{true} https://informationisbeautiful.net/visualizations/based-on-a-true-true-story/, visitado el 14/4/2020.


\bibitem{tab} https://www.tableau.com/solutions/, visitado el 4/5/2020.

\bibitem{public} https://public.tableau.com/, visitado el 25/5/2020.

\bibitem{github} https://github.com/, visitado el 25/5/2020.

\end{thebibliography}




\end{document}


